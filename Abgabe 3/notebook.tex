
% Default to the notebook output style

    


% Inherit from the specified cell style.




    
\documentclass[11pt]{article}

    
    
    \usepackage[T1]{fontenc}
    % Nicer default font (+ math font) than Computer Modern for most use cases
    \usepackage{mathpazo}

    % Basic figure setup, for now with no caption control since it's done
    % automatically by Pandoc (which extracts ![](path) syntax from Markdown).
    \usepackage{graphicx}
    % We will generate all images so they have a width \maxwidth. This means
    % that they will get their normal width if they fit onto the page, but
    % are scaled down if they would overflow the margins.
    \makeatletter
    \def\maxwidth{\ifdim\Gin@nat@width>\linewidth\linewidth
    \else\Gin@nat@width\fi}
    \makeatother
    \let\Oldincludegraphics\includegraphics
    % Set max figure width to be 80% of text width, for now hardcoded.
    \renewcommand{\includegraphics}[1]{\Oldincludegraphics[width=.8\maxwidth]{#1}}
    % Ensure that by default, figures have no caption (until we provide a
    % proper Figure object with a Caption API and a way to capture that
    % in the conversion process - todo).
    \usepackage{caption}
    \DeclareCaptionLabelFormat{nolabel}{}
    \captionsetup{labelformat=nolabel}

    \usepackage{adjustbox} % Used to constrain images to a maximum size 
    \usepackage{xcolor} % Allow colors to be defined
    \usepackage{enumerate} % Needed for markdown enumerations to work
    \usepackage{geometry} % Used to adjust the document margins
    \usepackage{amsmath} % Equations
    \usepackage{amssymb} % Equations
    \usepackage{textcomp} % defines textquotesingle
    % Hack from http://tex.stackexchange.com/a/47451/13684:
    \AtBeginDocument{%
        \def\PYZsq{\textquotesingle}% Upright quotes in Pygmentized code
    }
    \usepackage{upquote} % Upright quotes for verbatim code
    \usepackage{eurosym} % defines \euro
    \usepackage[mathletters]{ucs} % Extended unicode (utf-8) support
    \usepackage[utf8x]{inputenc} % Allow utf-8 characters in the tex document
    \usepackage{fancyvrb} % verbatim replacement that allows latex
    \usepackage{grffile} % extends the file name processing of package graphics 
                         % to support a larger range 
    % The hyperref package gives us a pdf with properly built
    % internal navigation ('pdf bookmarks' for the table of contents,
    % internal cross-reference links, web links for URLs, etc.)
    \usepackage{hyperref}
    \usepackage{longtable} % longtable support required by pandoc >1.10
    \usepackage{booktabs}  % table support for pandoc > 1.12.2
    \usepackage[inline]{enumitem} % IRkernel/repr support (it uses the enumerate* environment)
    \usepackage[normalem]{ulem} % ulem is needed to support strikethroughs (\sout)
                                % normalem makes italics be italics, not underlines
    

    
    
    % Colors for the hyperref package
    \definecolor{urlcolor}{rgb}{0,.145,.698}
    \definecolor{linkcolor}{rgb}{.71,0.21,0.01}
    \definecolor{citecolor}{rgb}{.12,.54,.11}

    % ANSI colors
    \definecolor{ansi-black}{HTML}{3E424D}
    \definecolor{ansi-black-intense}{HTML}{282C36}
    \definecolor{ansi-red}{HTML}{E75C58}
    \definecolor{ansi-red-intense}{HTML}{B22B31}
    \definecolor{ansi-green}{HTML}{00A250}
    \definecolor{ansi-green-intense}{HTML}{007427}
    \definecolor{ansi-yellow}{HTML}{DDB62B}
    \definecolor{ansi-yellow-intense}{HTML}{B27D12}
    \definecolor{ansi-blue}{HTML}{208FFB}
    \definecolor{ansi-blue-intense}{HTML}{0065CA}
    \definecolor{ansi-magenta}{HTML}{D160C4}
    \definecolor{ansi-magenta-intense}{HTML}{A03196}
    \definecolor{ansi-cyan}{HTML}{60C6C8}
    \definecolor{ansi-cyan-intense}{HTML}{258F8F}
    \definecolor{ansi-white}{HTML}{C5C1B4}
    \definecolor{ansi-white-intense}{HTML}{A1A6B2}

    % commands and environments needed by pandoc snippets
    % extracted from the output of `pandoc -s`
    \providecommand{\tightlist}{%
      \setlength{\itemsep}{0pt}\setlength{\parskip}{0pt}}
    \DefineVerbatimEnvironment{Highlighting}{Verbatim}{commandchars=\\\{\}}
    % Add ',fontsize=\small' for more characters per line
    \newenvironment{Shaded}{}{}
    \newcommand{\KeywordTok}[1]{\textcolor[rgb]{0.00,0.44,0.13}{\textbf{{#1}}}}
    \newcommand{\DataTypeTok}[1]{\textcolor[rgb]{0.56,0.13,0.00}{{#1}}}
    \newcommand{\DecValTok}[1]{\textcolor[rgb]{0.25,0.63,0.44}{{#1}}}
    \newcommand{\BaseNTok}[1]{\textcolor[rgb]{0.25,0.63,0.44}{{#1}}}
    \newcommand{\FloatTok}[1]{\textcolor[rgb]{0.25,0.63,0.44}{{#1}}}
    \newcommand{\CharTok}[1]{\textcolor[rgb]{0.25,0.44,0.63}{{#1}}}
    \newcommand{\StringTok}[1]{\textcolor[rgb]{0.25,0.44,0.63}{{#1}}}
    \newcommand{\CommentTok}[1]{\textcolor[rgb]{0.38,0.63,0.69}{\textit{{#1}}}}
    \newcommand{\OtherTok}[1]{\textcolor[rgb]{0.00,0.44,0.13}{{#1}}}
    \newcommand{\AlertTok}[1]{\textcolor[rgb]{1.00,0.00,0.00}{\textbf{{#1}}}}
    \newcommand{\FunctionTok}[1]{\textcolor[rgb]{0.02,0.16,0.49}{{#1}}}
    \newcommand{\RegionMarkerTok}[1]{{#1}}
    \newcommand{\ErrorTok}[1]{\textcolor[rgb]{1.00,0.00,0.00}{\textbf{{#1}}}}
    \newcommand{\NormalTok}[1]{{#1}}
    
    % Additional commands for more recent versions of Pandoc
    \newcommand{\ConstantTok}[1]{\textcolor[rgb]{0.53,0.00,0.00}{{#1}}}
    \newcommand{\SpecialCharTok}[1]{\textcolor[rgb]{0.25,0.44,0.63}{{#1}}}
    \newcommand{\VerbatimStringTok}[1]{\textcolor[rgb]{0.25,0.44,0.63}{{#1}}}
    \newcommand{\SpecialStringTok}[1]{\textcolor[rgb]{0.73,0.40,0.53}{{#1}}}
    \newcommand{\ImportTok}[1]{{#1}}
    \newcommand{\DocumentationTok}[1]{\textcolor[rgb]{0.73,0.13,0.13}{\textit{{#1}}}}
    \newcommand{\AnnotationTok}[1]{\textcolor[rgb]{0.38,0.63,0.69}{\textbf{\textit{{#1}}}}}
    \newcommand{\CommentVarTok}[1]{\textcolor[rgb]{0.38,0.63,0.69}{\textbf{\textit{{#1}}}}}
    \newcommand{\VariableTok}[1]{\textcolor[rgb]{0.10,0.09,0.49}{{#1}}}
    \newcommand{\ControlFlowTok}[1]{\textcolor[rgb]{0.00,0.44,0.13}{\textbf{{#1}}}}
    \newcommand{\OperatorTok}[1]{\textcolor[rgb]{0.40,0.40,0.40}{{#1}}}
    \newcommand{\BuiltInTok}[1]{{#1}}
    \newcommand{\ExtensionTok}[1]{{#1}}
    \newcommand{\PreprocessorTok}[1]{\textcolor[rgb]{0.74,0.48,0.00}{{#1}}}
    \newcommand{\AttributeTok}[1]{\textcolor[rgb]{0.49,0.56,0.16}{{#1}}}
    \newcommand{\InformationTok}[1]{\textcolor[rgb]{0.38,0.63,0.69}{\textbf{\textit{{#1}}}}}
    \newcommand{\WarningTok}[1]{\textcolor[rgb]{0.38,0.63,0.69}{\textbf{\textit{{#1}}}}}
    
    
    % Define a nice break command that doesn't care if a line doesn't already
    % exist.
    \def\br{\hspace*{\fill} \\* }
    % Math Jax compatability definitions
    \def\gt{>}
    \def\lt{<}
    % Document parameters
    \title{Abgabe 3}
    
    
    

    % Pygments definitions
    
\makeatletter
\def\PY@reset{\let\PY@it=\relax \let\PY@bf=\relax%
    \let\PY@ul=\relax \let\PY@tc=\relax%
    \let\PY@bc=\relax \let\PY@ff=\relax}
\def\PY@tok#1{\csname PY@tok@#1\endcsname}
\def\PY@toks#1+{\ifx\relax#1\empty\else%
    \PY@tok{#1}\expandafter\PY@toks\fi}
\def\PY@do#1{\PY@bc{\PY@tc{\PY@ul{%
    \PY@it{\PY@bf{\PY@ff{#1}}}}}}}
\def\PY#1#2{\PY@reset\PY@toks#1+\relax+\PY@do{#2}}

\expandafter\def\csname PY@tok@w\endcsname{\def\PY@tc##1{\textcolor[rgb]{0.73,0.73,0.73}{##1}}}
\expandafter\def\csname PY@tok@c\endcsname{\let\PY@it=\textit\def\PY@tc##1{\textcolor[rgb]{0.25,0.50,0.50}{##1}}}
\expandafter\def\csname PY@tok@cp\endcsname{\def\PY@tc##1{\textcolor[rgb]{0.74,0.48,0.00}{##1}}}
\expandafter\def\csname PY@tok@k\endcsname{\let\PY@bf=\textbf\def\PY@tc##1{\textcolor[rgb]{0.00,0.50,0.00}{##1}}}
\expandafter\def\csname PY@tok@kp\endcsname{\def\PY@tc##1{\textcolor[rgb]{0.00,0.50,0.00}{##1}}}
\expandafter\def\csname PY@tok@kt\endcsname{\def\PY@tc##1{\textcolor[rgb]{0.69,0.00,0.25}{##1}}}
\expandafter\def\csname PY@tok@o\endcsname{\def\PY@tc##1{\textcolor[rgb]{0.40,0.40,0.40}{##1}}}
\expandafter\def\csname PY@tok@ow\endcsname{\let\PY@bf=\textbf\def\PY@tc##1{\textcolor[rgb]{0.67,0.13,1.00}{##1}}}
\expandafter\def\csname PY@tok@nb\endcsname{\def\PY@tc##1{\textcolor[rgb]{0.00,0.50,0.00}{##1}}}
\expandafter\def\csname PY@tok@nf\endcsname{\def\PY@tc##1{\textcolor[rgb]{0.00,0.00,1.00}{##1}}}
\expandafter\def\csname PY@tok@nc\endcsname{\let\PY@bf=\textbf\def\PY@tc##1{\textcolor[rgb]{0.00,0.00,1.00}{##1}}}
\expandafter\def\csname PY@tok@nn\endcsname{\let\PY@bf=\textbf\def\PY@tc##1{\textcolor[rgb]{0.00,0.00,1.00}{##1}}}
\expandafter\def\csname PY@tok@ne\endcsname{\let\PY@bf=\textbf\def\PY@tc##1{\textcolor[rgb]{0.82,0.25,0.23}{##1}}}
\expandafter\def\csname PY@tok@nv\endcsname{\def\PY@tc##1{\textcolor[rgb]{0.10,0.09,0.49}{##1}}}
\expandafter\def\csname PY@tok@no\endcsname{\def\PY@tc##1{\textcolor[rgb]{0.53,0.00,0.00}{##1}}}
\expandafter\def\csname PY@tok@nl\endcsname{\def\PY@tc##1{\textcolor[rgb]{0.63,0.63,0.00}{##1}}}
\expandafter\def\csname PY@tok@ni\endcsname{\let\PY@bf=\textbf\def\PY@tc##1{\textcolor[rgb]{0.60,0.60,0.60}{##1}}}
\expandafter\def\csname PY@tok@na\endcsname{\def\PY@tc##1{\textcolor[rgb]{0.49,0.56,0.16}{##1}}}
\expandafter\def\csname PY@tok@nt\endcsname{\let\PY@bf=\textbf\def\PY@tc##1{\textcolor[rgb]{0.00,0.50,0.00}{##1}}}
\expandafter\def\csname PY@tok@nd\endcsname{\def\PY@tc##1{\textcolor[rgb]{0.67,0.13,1.00}{##1}}}
\expandafter\def\csname PY@tok@s\endcsname{\def\PY@tc##1{\textcolor[rgb]{0.73,0.13,0.13}{##1}}}
\expandafter\def\csname PY@tok@sd\endcsname{\let\PY@it=\textit\def\PY@tc##1{\textcolor[rgb]{0.73,0.13,0.13}{##1}}}
\expandafter\def\csname PY@tok@si\endcsname{\let\PY@bf=\textbf\def\PY@tc##1{\textcolor[rgb]{0.73,0.40,0.53}{##1}}}
\expandafter\def\csname PY@tok@se\endcsname{\let\PY@bf=\textbf\def\PY@tc##1{\textcolor[rgb]{0.73,0.40,0.13}{##1}}}
\expandafter\def\csname PY@tok@sr\endcsname{\def\PY@tc##1{\textcolor[rgb]{0.73,0.40,0.53}{##1}}}
\expandafter\def\csname PY@tok@ss\endcsname{\def\PY@tc##1{\textcolor[rgb]{0.10,0.09,0.49}{##1}}}
\expandafter\def\csname PY@tok@sx\endcsname{\def\PY@tc##1{\textcolor[rgb]{0.00,0.50,0.00}{##1}}}
\expandafter\def\csname PY@tok@m\endcsname{\def\PY@tc##1{\textcolor[rgb]{0.40,0.40,0.40}{##1}}}
\expandafter\def\csname PY@tok@gh\endcsname{\let\PY@bf=\textbf\def\PY@tc##1{\textcolor[rgb]{0.00,0.00,0.50}{##1}}}
\expandafter\def\csname PY@tok@gu\endcsname{\let\PY@bf=\textbf\def\PY@tc##1{\textcolor[rgb]{0.50,0.00,0.50}{##1}}}
\expandafter\def\csname PY@tok@gd\endcsname{\def\PY@tc##1{\textcolor[rgb]{0.63,0.00,0.00}{##1}}}
\expandafter\def\csname PY@tok@gi\endcsname{\def\PY@tc##1{\textcolor[rgb]{0.00,0.63,0.00}{##1}}}
\expandafter\def\csname PY@tok@gr\endcsname{\def\PY@tc##1{\textcolor[rgb]{1.00,0.00,0.00}{##1}}}
\expandafter\def\csname PY@tok@ge\endcsname{\let\PY@it=\textit}
\expandafter\def\csname PY@tok@gs\endcsname{\let\PY@bf=\textbf}
\expandafter\def\csname PY@tok@gp\endcsname{\let\PY@bf=\textbf\def\PY@tc##1{\textcolor[rgb]{0.00,0.00,0.50}{##1}}}
\expandafter\def\csname PY@tok@go\endcsname{\def\PY@tc##1{\textcolor[rgb]{0.53,0.53,0.53}{##1}}}
\expandafter\def\csname PY@tok@gt\endcsname{\def\PY@tc##1{\textcolor[rgb]{0.00,0.27,0.87}{##1}}}
\expandafter\def\csname PY@tok@err\endcsname{\def\PY@bc##1{\setlength{\fboxsep}{0pt}\fcolorbox[rgb]{1.00,0.00,0.00}{1,1,1}{\strut ##1}}}
\expandafter\def\csname PY@tok@kc\endcsname{\let\PY@bf=\textbf\def\PY@tc##1{\textcolor[rgb]{0.00,0.50,0.00}{##1}}}
\expandafter\def\csname PY@tok@kd\endcsname{\let\PY@bf=\textbf\def\PY@tc##1{\textcolor[rgb]{0.00,0.50,0.00}{##1}}}
\expandafter\def\csname PY@tok@kn\endcsname{\let\PY@bf=\textbf\def\PY@tc##1{\textcolor[rgb]{0.00,0.50,0.00}{##1}}}
\expandafter\def\csname PY@tok@kr\endcsname{\let\PY@bf=\textbf\def\PY@tc##1{\textcolor[rgb]{0.00,0.50,0.00}{##1}}}
\expandafter\def\csname PY@tok@bp\endcsname{\def\PY@tc##1{\textcolor[rgb]{0.00,0.50,0.00}{##1}}}
\expandafter\def\csname PY@tok@fm\endcsname{\def\PY@tc##1{\textcolor[rgb]{0.00,0.00,1.00}{##1}}}
\expandafter\def\csname PY@tok@vc\endcsname{\def\PY@tc##1{\textcolor[rgb]{0.10,0.09,0.49}{##1}}}
\expandafter\def\csname PY@tok@vg\endcsname{\def\PY@tc##1{\textcolor[rgb]{0.10,0.09,0.49}{##1}}}
\expandafter\def\csname PY@tok@vi\endcsname{\def\PY@tc##1{\textcolor[rgb]{0.10,0.09,0.49}{##1}}}
\expandafter\def\csname PY@tok@vm\endcsname{\def\PY@tc##1{\textcolor[rgb]{0.10,0.09,0.49}{##1}}}
\expandafter\def\csname PY@tok@sa\endcsname{\def\PY@tc##1{\textcolor[rgb]{0.73,0.13,0.13}{##1}}}
\expandafter\def\csname PY@tok@sb\endcsname{\def\PY@tc##1{\textcolor[rgb]{0.73,0.13,0.13}{##1}}}
\expandafter\def\csname PY@tok@sc\endcsname{\def\PY@tc##1{\textcolor[rgb]{0.73,0.13,0.13}{##1}}}
\expandafter\def\csname PY@tok@dl\endcsname{\def\PY@tc##1{\textcolor[rgb]{0.73,0.13,0.13}{##1}}}
\expandafter\def\csname PY@tok@s2\endcsname{\def\PY@tc##1{\textcolor[rgb]{0.73,0.13,0.13}{##1}}}
\expandafter\def\csname PY@tok@sh\endcsname{\def\PY@tc##1{\textcolor[rgb]{0.73,0.13,0.13}{##1}}}
\expandafter\def\csname PY@tok@s1\endcsname{\def\PY@tc##1{\textcolor[rgb]{0.73,0.13,0.13}{##1}}}
\expandafter\def\csname PY@tok@mb\endcsname{\def\PY@tc##1{\textcolor[rgb]{0.40,0.40,0.40}{##1}}}
\expandafter\def\csname PY@tok@mf\endcsname{\def\PY@tc##1{\textcolor[rgb]{0.40,0.40,0.40}{##1}}}
\expandafter\def\csname PY@tok@mh\endcsname{\def\PY@tc##1{\textcolor[rgb]{0.40,0.40,0.40}{##1}}}
\expandafter\def\csname PY@tok@mi\endcsname{\def\PY@tc##1{\textcolor[rgb]{0.40,0.40,0.40}{##1}}}
\expandafter\def\csname PY@tok@il\endcsname{\def\PY@tc##1{\textcolor[rgb]{0.40,0.40,0.40}{##1}}}
\expandafter\def\csname PY@tok@mo\endcsname{\def\PY@tc##1{\textcolor[rgb]{0.40,0.40,0.40}{##1}}}
\expandafter\def\csname PY@tok@ch\endcsname{\let\PY@it=\textit\def\PY@tc##1{\textcolor[rgb]{0.25,0.50,0.50}{##1}}}
\expandafter\def\csname PY@tok@cm\endcsname{\let\PY@it=\textit\def\PY@tc##1{\textcolor[rgb]{0.25,0.50,0.50}{##1}}}
\expandafter\def\csname PY@tok@cpf\endcsname{\let\PY@it=\textit\def\PY@tc##1{\textcolor[rgb]{0.25,0.50,0.50}{##1}}}
\expandafter\def\csname PY@tok@c1\endcsname{\let\PY@it=\textit\def\PY@tc##1{\textcolor[rgb]{0.25,0.50,0.50}{##1}}}
\expandafter\def\csname PY@tok@cs\endcsname{\let\PY@it=\textit\def\PY@tc##1{\textcolor[rgb]{0.25,0.50,0.50}{##1}}}

\def\PYZbs{\char`\\}
\def\PYZus{\char`\_}
\def\PYZob{\char`\{}
\def\PYZcb{\char`\}}
\def\PYZca{\char`\^}
\def\PYZam{\char`\&}
\def\PYZlt{\char`\<}
\def\PYZgt{\char`\>}
\def\PYZsh{\char`\#}
\def\PYZpc{\char`\%}
\def\PYZdl{\char`\$}
\def\PYZhy{\char`\-}
\def\PYZsq{\char`\'}
\def\PYZdq{\char`\"}
\def\PYZti{\char`\~}
% for compatibility with earlier versions
\def\PYZat{@}
\def\PYZlb{[}
\def\PYZrb{]}
\makeatother


    % Exact colors from NB
    \definecolor{incolor}{rgb}{0.0, 0.0, 0.5}
    \definecolor{outcolor}{rgb}{0.545, 0.0, 0.0}



    
    % Prevent overflowing lines due to hard-to-break entities
    \sloppy 
    % Setup hyperref package
    \hypersetup{
      breaklinks=true,  % so long urls are correctly broken across lines
      colorlinks=true,
      urlcolor=urlcolor,
      linkcolor=linkcolor,
      citecolor=citecolor,
      }
    % Slightly bigger margins than the latex defaults
    
    \geometry{verbose,tmargin=1in,bmargin=1in,lmargin=1in,rmargin=1in}
    
    

    \begin{document}
    
    
    \maketitle
    
    

    
    \section{Abgabe 3: Schwingungen}\label{abgabe-3-schwingungen}

\textbf{Teammitglieder:} René Zarwel, Matthias Kastenmüller

Importieren allgemein benötigter Bibliotheken:

    \begin{Verbatim}[commandchars=\\\{\}]
{\color{incolor}In [{\color{incolor}1}]:} \PY{k+kn}{import} \PY{n+nn}{matplotlib}\PY{n+nn}{.}\PY{n+nn}{pyplot} \PY{k}{as} \PY{n+nn}{plt}
        \PY{k+kn}{import} \PY{n+nn}{matplotlib}\PY{n+nn}{.}\PY{n+nn}{patches} \PY{k}{as} \PY{n+nn}{mpatches}
        \PY{k+kn}{import} \PY{n+nn}{pylab} \PY{k}{as} \PY{n+nn}{pylab}
        \PY{k+kn}{import} \PY{n+nn}{numpy}\PY{n+nn}{.}\PY{n+nn}{polynomial}\PY{n+nn}{.}\PY{n+nn}{polynomial} \PY{k}{as} \PY{n+nn}{poly}
        \PY{k+kn}{import} \PY{n+nn}{numpy} \PY{k}{as} \PY{n+nn}{np}
        \PY{k+kn}{from} \PY{n+nn}{scipy}\PY{n+nn}{.}\PY{n+nn}{integrate} \PY{k}{import} \PY{n}{odeint}
        \PY{k+kn}{import} \PY{n+nn}{math}
        \PY{k+kn}{from} \PY{n+nn}{numpy} \PY{k}{import} \PY{n}{array}\PY{p}{,} \PY{n}{polyval}\PY{p}{,} \PY{n}{polyfit}
        \PY{k+kn}{import} \PY{n+nn}{vpython} \PY{k}{as} \PY{n+nn}{vp}         \PY{c+c1}{\PYZsh{} get VPython modules for animation}
        \PY{n}{vec}\PY{o}{=}\PY{n}{vp}\PY{o}{.}\PY{n}{vector}
\end{Verbatim}


    
    \begin{verbatim}
<IPython.core.display.Javascript object>
    \end{verbatim}

    
    
    \begin{verbatim}
<IPython.core.display.Javascript object>
    \end{verbatim}

    
    
    \begin{verbatim}
<IPython.core.display.Javascript object>
    \end{verbatim}

    
    
    \begin{verbatim}
<IPython.core.display.Javascript object>
    \end{verbatim}

    
    
    \begin{verbatim}
<IPython.core.display.Javascript object>
    \end{verbatim}

    
    
    \begin{verbatim}
<IPython.core.display.Javascript object>
    \end{verbatim}

    
    
    \begin{verbatim}
<IPython.core.display.HTML object>
    \end{verbatim}

    
    
    \begin{verbatim}
<IPython.core.display.Javascript object>
    \end{verbatim}

    
    \subsection{Aufgabe 1: Gedämpfte
Schwingungen}\label{aufgabe-1-geduxe4mpfte-schwingungen}

    Bei einer harmonischen Schwingung wirkt lediglich die rücktreibende
Kraft. Daraus ergibt sich folgenden Schwingungsgleichung:

\[
F = -k \cdot x \quad \text{mit} \quad k=\text{Federkonstante} \\
\omega_0 = \sqrt{\frac{k}{m}}  \quad \text{- Ungedämpfte Eigenkreisfrequenz } \\
\ddot{x} + \omega_0^2 \cdot x = 0
\]

Bei der gedämpften Schwingung wirk zusätzlich zur rücktreibenden Kraft
noch eine Reibungskraft \(F_R\). So wird die Schwingungsgleichung
folgendermaßen erweitert:

\[
F_R = - c \cdot v^n \quad\text{mit}\quad c=\text{Dämpfungskonstante} \quad\text{und}\quad n=\{0,\frac{1}{2}, 1, 2\} \\
\gamma = \frac{c}{2 \cdot m} \quad \text{- Abklingkonstante} \\
\ddot{x} + 2 \cdot \gamma \cdot \dot{x}^n + \omega_0^2 \cdot x = 0
\]

Zur Lösung dieser Differentialgleichung stellen wir folgende Gleichungen
auf:

\[
y_1 = x \\
y_2 = \dot{x} \\
\dot{y_1} = y_2 = \dot{x} \\
\dot{y_2} = \ddot{x} = - 2 \cdot \gamma \cdot \dot{x}^n - \omega_0^2 \cdot x
\]

Durch die Bestimmung vom Anfangswerten für \(x\) und \(\dot{x}=v\)
(Anfangswertproblem) können diese nun mittels der ODEINT-Funktion
numerisch gelöst werden. Hierfür wurde definiert: \[
x = 1 \\
\dot{x}=0
\]

    \begin{Verbatim}[commandchars=\\\{\}]
{\color{incolor}In [{\color{incolor}2}]:} \PY{n}{x0} \PY{o}{=} \PY{p}{[}\PY{l+m+mi}{1}\PY{p}{,} \PY{l+m+mi}{0}\PY{p}{]}          \PY{c+c1}{\PYZsh{} Anfangsbedingungen [x, v]}
        \PY{n}{t} \PY{o}{=} \PY{n}{np}\PY{o}{.}\PY{n}{linspace}\PY{p}{(}\PY{l+m+mi}{0}\PY{p}{,}\PY{l+m+mi}{10}\PY{p}{,}\PY{l+m+mi}{500}\PY{p}{)}  \PY{c+c1}{\PYZsh{} 500 Zeitschritte von 0 bis 10s}
        
        \PY{c+c1}{\PYZsh{}Own Sign Func cause np.sign is unstable with odeint}
        \PY{n}{sgn} \PY{o}{=} \PY{k}{lambda} \PY{n}{x}\PY{p}{:} \PY{n}{np}\PY{o}{.}\PY{n}{tanh}\PY{p}{(}\PY{l+m+mi}{100}\PY{o}{*}\PY{n}{x}\PY{p}{)}
        
        \PY{c+c1}{\PYZsh{}Gleichungssystem für die Differentialgleichung als Funktion}
        \PY{k}{def} \PY{n+nf}{swing}\PY{p}{(}\PY{n}{x}\PY{p}{,}\PY{n}{t}\PY{p}{)}\PY{p}{:}
            \PY{k}{return}\PY{p}{[} \PY{n}{x}\PY{p}{[}\PY{l+m+mi}{1}\PY{p}{]}\PY{p}{,} \PY{o}{\PYZhy{}}\PY{l+m+mi}{2}\PY{o}{*}\PY{n}{gamma}\PY{o}{*}\PY{n}{sgn}\PY{p}{(}\PY{n}{x}\PY{p}{[}\PY{l+m+mi}{1}\PY{p}{]}\PY{p}{)}\PY{o}{*}\PY{n+nb}{abs}\PY{p}{(}\PY{n}{x}\PY{p}{[}\PY{l+m+mi}{1}\PY{p}{]}\PY{p}{)}\PY{o}{*}\PY{o}{*}\PY{n}{n}\PY{o}{\PYZhy{}}\PY{p}{(}\PY{n}{omega0}\PY{o}{*}\PY{o}{*}\PY{l+m+mi}{2}\PY{p}{)}\PY{o}{*}\PY{n}{x}\PY{p}{[}\PY{l+m+mi}{0}\PY{p}{]} \PY{p}{]}
        
        \PY{c+c1}{\PYZsh{}Definiton von Konstanten für Schwingfall}
        \PY{n}{omega0} \PY{o}{=} \PY{l+m+mi}{6}                  
        \PY{n}{gamma} \PY{o}{=} \PY{l+m+mi}{1}
        
        \PY{c+c1}{\PYZsh{}Lösung und Plot für n=(0, 1/2, 1, 2)}
        \PY{n}{n}\PY{o}{=}\PY{l+m+mi}{0}
        \PY{n}{x} \PY{o}{=} \PY{n}{odeint}\PY{p}{(}\PY{n}{func}\PY{o}{=}\PY{n}{swing}\PY{p}{,} \PY{n}{y0}\PY{o}{=}\PY{n}{x0}\PY{p}{,} \PY{n}{t}\PY{o}{=}\PY{n}{t}\PY{p}{)}
        \PY{n}{plt}\PY{o}{.}\PY{n}{plot}\PY{p}{(}\PY{n}{t}\PY{p}{,} \PY{n}{x}\PY{p}{[}\PY{p}{:}\PY{p}{,}\PY{l+m+mi}{0}\PY{p}{]}\PY{p}{,} \PY{n}{label}\PY{o}{=}\PY{l+s+s2}{\PYZdq{}}\PY{l+s+s2}{n=0}\PY{l+s+s2}{\PYZdq{}}\PY{p}{)}
        
        \PY{n}{n}\PY{o}{=}\PY{l+m+mi}{1}\PY{o}{/}\PY{l+m+mi}{2}
        \PY{n}{x} \PY{o}{=} \PY{n}{odeint}\PY{p}{(}\PY{n}{func}\PY{o}{=}\PY{n}{swing}\PY{p}{,} \PY{n}{y0}\PY{o}{=}\PY{n}{x0}\PY{p}{,} \PY{n}{t}\PY{o}{=}\PY{n}{t}\PY{p}{)}
        \PY{n}{plt}\PY{o}{.}\PY{n}{plot}\PY{p}{(}\PY{n}{t}\PY{p}{,} \PY{n}{x}\PY{p}{[}\PY{p}{:}\PY{p}{,}\PY{l+m+mi}{0}\PY{p}{]}\PY{p}{,} \PY{n}{label}\PY{o}{=}\PY{l+s+s2}{\PYZdq{}}\PY{l+s+s2}{n=1/2}\PY{l+s+s2}{\PYZdq{}}\PY{p}{)}
        
        \PY{n}{n}\PY{o}{=}\PY{l+m+mi}{1}
        \PY{n}{x} \PY{o}{=} \PY{n}{odeint}\PY{p}{(}\PY{n}{func}\PY{o}{=}\PY{n}{swing}\PY{p}{,} \PY{n}{y0}\PY{o}{=}\PY{n}{x0}\PY{p}{,} \PY{n}{t}\PY{o}{=}\PY{n}{t}\PY{p}{)}
        \PY{n}{plt}\PY{o}{.}\PY{n}{plot}\PY{p}{(}\PY{n}{t}\PY{p}{,} \PY{n}{x}\PY{p}{[}\PY{p}{:}\PY{p}{,}\PY{l+m+mi}{0}\PY{p}{]}\PY{p}{,} \PY{n}{label}\PY{o}{=}\PY{l+s+s2}{\PYZdq{}}\PY{l+s+s2}{n=1}\PY{l+s+s2}{\PYZdq{}}\PY{p}{)}
        
        \PY{n}{n}\PY{o}{=}\PY{l+m+mi}{2}
        \PY{n}{x} \PY{o}{=} \PY{n}{odeint}\PY{p}{(}\PY{n}{func}\PY{o}{=}\PY{n}{swing}\PY{p}{,} \PY{n}{y0}\PY{o}{=}\PY{n}{x0}\PY{p}{,} \PY{n}{t}\PY{o}{=}\PY{n}{t}\PY{p}{)}
        \PY{n}{plt}\PY{o}{.}\PY{n}{plot}\PY{p}{(}\PY{n}{t}\PY{p}{,} \PY{n}{x}\PY{p}{[}\PY{p}{:}\PY{p}{,}\PY{l+m+mi}{0}\PY{p}{]}\PY{p}{,} \PY{n}{label}\PY{o}{=}\PY{l+s+s2}{\PYZdq{}}\PY{l+s+s2}{n=2}\PY{l+s+s2}{\PYZdq{}}\PY{p}{)}
        
        \PY{c+c1}{\PYZsh{} Plot Beschriftung}
        \PY{n}{plt}\PY{o}{.}\PY{n}{title}\PY{p}{(}\PY{l+s+s1}{\PYZsq{}}\PY{l+s+s1}{Weg/Zeit\PYZhy{}Diagramm}\PY{l+s+s1}{\PYZsq{}}\PY{p}{)}
        \PY{n}{plt}\PY{o}{.}\PY{n}{xlabel}\PY{p}{(}\PY{l+s+s1}{\PYZsq{}}\PY{l+s+s1}{Zeit (t)}\PY{l+s+s1}{\PYZsq{}}\PY{p}{)}
        \PY{n}{plt}\PY{o}{.}\PY{n}{ylabel}\PY{p}{(}\PY{l+s+s1}{\PYZsq{}}\PY{l+s+s1}{Weg (Theta)}\PY{l+s+s1}{\PYZsq{}}\PY{p}{)}
        \PY{n}{plt}\PY{o}{.}\PY{n}{legend}\PY{p}{(}\PY{p}{)}
\end{Verbatim}


\begin{Verbatim}[commandchars=\\\{\}]
{\color{outcolor}Out[{\color{outcolor}2}]:} <matplotlib.legend.Legend at 0x1154cb8d0>
\end{Verbatim}
            
    \begin{center}
    \adjustimage{max size={0.9\linewidth}{0.9\paperheight}}{output_4_1.png}
    \end{center}
    { \hspace*{\fill} \\}
    
    Im obigen Plot ist der Schwingfall dargestellt, bei dem die ungedämpfte
Eigenkreisfrequenz größer als die Dämpfungskonstante ist. Man erkennt,
dass bei konstanter Dämpfung ( \(n=0\) ) die Schwingung auch konstant
abnimmt. Bei den anderen Fällen nimmt die Schwingung stärker ab und
gleicht einer Parabel ( \$ n=\frac{1}{2}\$ ), einer e-Funktion ( \(n=1\)
) und einer Hyperbel ( \(n=2\) ). Besonders interessant ist auch der
Fall der zur Geschwindigkeit quadratischen Dämpfung, da dies durch
Umwirblungen noch lange weiter schwingt.

    \begin{Verbatim}[commandchars=\\\{\}]
{\color{incolor}In [{\color{incolor}3}]:} \PY{c+c1}{\PYZsh{}Definiton von Konstanten für aperiodischen Grenzfall}
        \PY{n}{omega0} \PY{o}{=} \PY{l+m+mi}{6}                  
        \PY{n}{gamma} \PY{o}{=} \PY{l+m+mi}{6}
        
        \PY{c+c1}{\PYZsh{}Lösung und Plot für n=(0, 1/2, 1, 2)}
        
        \PY{n}{n}\PY{o}{=}\PY{l+m+mi}{0}
        \PY{n}{x} \PY{o}{=} \PY{n}{odeint}\PY{p}{(}\PY{n}{func}\PY{o}{=}\PY{n}{swing}\PY{p}{,} \PY{n}{y0}\PY{o}{=}\PY{n}{x0}\PY{p}{,} \PY{n}{t}\PY{o}{=}\PY{n}{t}\PY{p}{)}
        \PY{n}{plt}\PY{o}{.}\PY{n}{plot}\PY{p}{(}\PY{n}{t}\PY{p}{,} \PY{n}{x}\PY{p}{[}\PY{p}{:}\PY{p}{,}\PY{l+m+mi}{0}\PY{p}{]}\PY{p}{,} \PY{n}{label}\PY{o}{=}\PY{l+s+s2}{\PYZdq{}}\PY{l+s+s2}{n=0}\PY{l+s+s2}{\PYZdq{}}\PY{p}{)}
        
        \PY{n}{n}\PY{o}{=}\PY{l+m+mi}{1}\PY{o}{/}\PY{l+m+mi}{2}
        \PY{n}{x} \PY{o}{=} \PY{n}{odeint}\PY{p}{(}\PY{n}{func}\PY{o}{=}\PY{n}{swing}\PY{p}{,} \PY{n}{y0}\PY{o}{=}\PY{n}{x0}\PY{p}{,} \PY{n}{t}\PY{o}{=}\PY{n}{t}\PY{p}{)}
        \PY{n}{plt}\PY{o}{.}\PY{n}{plot}\PY{p}{(}\PY{n}{t}\PY{p}{,} \PY{n}{x}\PY{p}{[}\PY{p}{:}\PY{p}{,}\PY{l+m+mi}{0}\PY{p}{]}\PY{p}{,} \PY{n}{label}\PY{o}{=}\PY{l+s+s2}{\PYZdq{}}\PY{l+s+s2}{n=1/2}\PY{l+s+s2}{\PYZdq{}}\PY{p}{)}
        
        \PY{n}{n}\PY{o}{=}\PY{l+m+mi}{1}
        \PY{n}{x} \PY{o}{=} \PY{n}{odeint}\PY{p}{(}\PY{n}{func}\PY{o}{=}\PY{n}{swing}\PY{p}{,} \PY{n}{y0}\PY{o}{=}\PY{n}{x0}\PY{p}{,} \PY{n}{t}\PY{o}{=}\PY{n}{t}\PY{p}{)}
        \PY{n}{plt}\PY{o}{.}\PY{n}{plot}\PY{p}{(}\PY{n}{t}\PY{p}{,} \PY{n}{x}\PY{p}{[}\PY{p}{:}\PY{p}{,}\PY{l+m+mi}{0}\PY{p}{]}\PY{p}{,} \PY{n}{label}\PY{o}{=}\PY{l+s+s2}{\PYZdq{}}\PY{l+s+s2}{n=1}\PY{l+s+s2}{\PYZdq{}}\PY{p}{)}
        
        \PY{n}{n}\PY{o}{=}\PY{l+m+mi}{2}
        \PY{n}{x} \PY{o}{=} \PY{n}{odeint}\PY{p}{(}\PY{n}{func}\PY{o}{=}\PY{n}{swing}\PY{p}{,} \PY{n}{y0}\PY{o}{=}\PY{n}{x0}\PY{p}{,} \PY{n}{t}\PY{o}{=}\PY{n}{t}\PY{p}{)}
        \PY{n}{plt}\PY{o}{.}\PY{n}{plot}\PY{p}{(}\PY{n}{t}\PY{p}{,} \PY{n}{x}\PY{p}{[}\PY{p}{:}\PY{p}{,}\PY{l+m+mi}{0}\PY{p}{]}\PY{p}{,} \PY{n}{label}\PY{o}{=}\PY{l+s+s2}{\PYZdq{}}\PY{l+s+s2}{n=2}\PY{l+s+s2}{\PYZdq{}}\PY{p}{)}
        
        \PY{c+c1}{\PYZsh{} Plot Beschriftung}
        \PY{n}{plt}\PY{o}{.}\PY{n}{title}\PY{p}{(}\PY{l+s+s1}{\PYZsq{}}\PY{l+s+s1}{Weg/Zeit\PYZhy{}Diagramm}\PY{l+s+s1}{\PYZsq{}}\PY{p}{)}
        \PY{n}{plt}\PY{o}{.}\PY{n}{xlabel}\PY{p}{(}\PY{l+s+s1}{\PYZsq{}}\PY{l+s+s1}{Zeit (t)}\PY{l+s+s1}{\PYZsq{}}\PY{p}{)}
        \PY{n}{plt}\PY{o}{.}\PY{n}{ylabel}\PY{p}{(}\PY{l+s+s1}{\PYZsq{}}\PY{l+s+s1}{Weg (Theta)}\PY{l+s+s1}{\PYZsq{}}\PY{p}{)}
        \PY{n}{plt}\PY{o}{.}\PY{n}{legend}\PY{p}{(}\PY{p}{)}
\end{Verbatim}


\begin{Verbatim}[commandchars=\\\{\}]
{\color{outcolor}Out[{\color{outcolor}3}]:} <matplotlib.legend.Legend at 0x1155f4518>
\end{Verbatim}
            
    \begin{center}
    \adjustimage{max size={0.9\linewidth}{0.9\paperheight}}{output_6_1.png}
    \end{center}
    { \hspace*{\fill} \\}
    
    Im obigen Plot ist der aperiodische Grenzfall dargestellt, bei dem die
ungedämpfte Eigenkreisfrequenz gleich der Dämpfungskonstante ist. In
diesem Fall wirkt die Dämpfung so, dass der Ruhezustand möglichst
schnell erreicht wird. Dies sieht man an einer schnellen Annäherung an
\(0\). Lediglich bei \(n=2\) besteht noch ein Nachschwingen ähnlich zum
vorherigen Beispiel. Interessant ist die konstante Dämpfung. Durch die
im Verhältnis zur Dämpfung hohe Startposition \(x=1\) kann das
herabfallende Objekt nicht optimal gedämpft werden, sodass es sich
zunächst an der 0 Position vorbei bewegt. Anschließend bleibt es bei
einer ausgeglichenen Position stehen und bewegt sich langsam auf die 0
Position zu.

    \begin{Verbatim}[commandchars=\\\{\}]
{\color{incolor}In [{\color{incolor}4}]:} \PY{c+c1}{\PYZsh{}Definiton von Konstanten für Kriechfall}
        \PY{n}{omega0} \PY{o}{=} \PY{l+m+mi}{6}                  
        \PY{n}{gamma} \PY{o}{=} \PY{l+m+mi}{12}
        
        \PY{c+c1}{\PYZsh{}Lösung und Plot für n=(0, 1/2, 1, 2)}
        
        \PY{n}{n}\PY{o}{=}\PY{l+m+mi}{0}
        \PY{n}{x} \PY{o}{=} \PY{n}{odeint}\PY{p}{(}\PY{n}{func}\PY{o}{=}\PY{n}{swing}\PY{p}{,} \PY{n}{y0}\PY{o}{=}\PY{n}{x0}\PY{p}{,} \PY{n}{t}\PY{o}{=}\PY{n}{t}\PY{p}{,} \PY{n}{atol}\PY{o}{=}\PY{l+m+mf}{1e\PYZhy{}1}\PY{p}{)}
        \PY{n}{plt}\PY{o}{.}\PY{n}{plot}\PY{p}{(}\PY{n}{t}\PY{p}{,} \PY{n}{x}\PY{p}{[}\PY{p}{:}\PY{p}{,}\PY{l+m+mi}{0}\PY{p}{]}\PY{p}{,} \PY{n}{label}\PY{o}{=}\PY{l+s+s2}{\PYZdq{}}\PY{l+s+s2}{n=0}\PY{l+s+s2}{\PYZdq{}}\PY{p}{)}
        
        \PY{n}{n}\PY{o}{=}\PY{l+m+mi}{1}\PY{o}{/}\PY{l+m+mi}{2}
        \PY{n}{x} \PY{o}{=} \PY{n}{odeint}\PY{p}{(}\PY{n}{func}\PY{o}{=}\PY{n}{swing}\PY{p}{,} \PY{n}{y0}\PY{o}{=}\PY{n}{x0}\PY{p}{,} \PY{n}{t}\PY{o}{=}\PY{n}{t}\PY{p}{)}
        \PY{n}{plt}\PY{o}{.}\PY{n}{plot}\PY{p}{(}\PY{n}{t}\PY{p}{,} \PY{n}{x}\PY{p}{[}\PY{p}{:}\PY{p}{,}\PY{l+m+mi}{0}\PY{p}{]}\PY{p}{,} \PY{n}{label}\PY{o}{=}\PY{l+s+s2}{\PYZdq{}}\PY{l+s+s2}{n=1/2}\PY{l+s+s2}{\PYZdq{}}\PY{p}{)}
        
        \PY{n}{n}\PY{o}{=}\PY{l+m+mi}{1}
        \PY{n}{x} \PY{o}{=} \PY{n}{odeint}\PY{p}{(}\PY{n}{func}\PY{o}{=}\PY{n}{swing}\PY{p}{,} \PY{n}{y0}\PY{o}{=}\PY{n}{x0}\PY{p}{,} \PY{n}{t}\PY{o}{=}\PY{n}{t}\PY{p}{)}
        \PY{n}{plt}\PY{o}{.}\PY{n}{plot}\PY{p}{(}\PY{n}{t}\PY{p}{,} \PY{n}{x}\PY{p}{[}\PY{p}{:}\PY{p}{,}\PY{l+m+mi}{0}\PY{p}{]}\PY{p}{,} \PY{n}{label}\PY{o}{=}\PY{l+s+s2}{\PYZdq{}}\PY{l+s+s2}{n=1}\PY{l+s+s2}{\PYZdq{}}\PY{p}{)}
        
        \PY{n}{n}\PY{o}{=}\PY{l+m+mi}{2}
        \PY{n}{x} \PY{o}{=} \PY{n}{odeint}\PY{p}{(}\PY{n}{func}\PY{o}{=}\PY{n}{swing}\PY{p}{,} \PY{n}{y0}\PY{o}{=}\PY{n}{x0}\PY{p}{,} \PY{n}{t}\PY{o}{=}\PY{n}{t}\PY{p}{)}
        \PY{n}{plt}\PY{o}{.}\PY{n}{plot}\PY{p}{(}\PY{n}{t}\PY{p}{,} \PY{n}{x}\PY{p}{[}\PY{p}{:}\PY{p}{,}\PY{l+m+mi}{0}\PY{p}{]}\PY{p}{,} \PY{n}{label}\PY{o}{=}\PY{l+s+s2}{\PYZdq{}}\PY{l+s+s2}{n=2}\PY{l+s+s2}{\PYZdq{}}\PY{p}{)}
        
        \PY{c+c1}{\PYZsh{} Plot Beschriftung}
        \PY{n}{plt}\PY{o}{.}\PY{n}{title}\PY{p}{(}\PY{l+s+s1}{\PYZsq{}}\PY{l+s+s1}{Weg/Zeit\PYZhy{}Diagramm}\PY{l+s+s1}{\PYZsq{}}\PY{p}{)}
        \PY{n}{plt}\PY{o}{.}\PY{n}{xlabel}\PY{p}{(}\PY{l+s+s1}{\PYZsq{}}\PY{l+s+s1}{Zeit (t)}\PY{l+s+s1}{\PYZsq{}}\PY{p}{)}
        \PY{n}{plt}\PY{o}{.}\PY{n}{ylabel}\PY{p}{(}\PY{l+s+s1}{\PYZsq{}}\PY{l+s+s1}{Weg (Theta)}\PY{l+s+s1}{\PYZsq{}}\PY{p}{)}
        \PY{n}{plt}\PY{o}{.}\PY{n}{legend}\PY{p}{(}\PY{p}{)}
\end{Verbatim}


\begin{Verbatim}[commandchars=\\\{\}]
{\color{outcolor}Out[{\color{outcolor}4}]:} <matplotlib.legend.Legend at 0x1156dd6a0>
\end{Verbatim}
            
    \begin{center}
    \adjustimage{max size={0.9\linewidth}{0.9\paperheight}}{output_8_1.png}
    \end{center}
    { \hspace*{\fill} \\}
    
    Im obigen Plot ist der Kriechfall dargestellt, bei dem die ungedämpfte
Eigenkreisfrequenz kleiner als die Dämpfungskonstante ist. Dies ist
daran ersichtlich, dass sich das Objekt durch die starke Dämpfung
langsamer als im vorherigen Beispiel auf die 0 Position zu bewegt. Auch
hier schwingt im Fall von \(n=2\) das Object noch etwas nach und die
konstante Dämpfung findet einen Punkt, ab dem sich das Objekt nur noch
sehr langsam weiter bewegt. Bei diesem Punkt ist die treibende Kraft der
konstanten Dämpfung ausgeglichen.

    \subsection{Aufgabe 2: Das nichtlineare Pendel
I}\label{aufgabe-2-das-nichtlineare-pendel-i}

    Es soll folgenden nicht Differentialgleichung für ein nichtlineares
Pendel gelöst werden:

\[
\ddot{\theta} + \frac{g}{l} \cdot \sin(\theta) = 0 \\
\text{mit} \quad g=9.81\frac{m}{s^2} \quad \text{und} \quad l=10cm
\]

Ähnlich zur Aufgabe 1 können für die Lösung folgende Gleichungen
aufgestellt werden.

\[
y_1 = x \\
y_2 = \dot{x} \\
\dot{y_1} = y_2 = \dot{x} \\
\dot{y_2} = \ddot{x} = -\frac{g}{l} \cdot \sin(\theta)
\]

    \begin{Verbatim}[commandchars=\\\{\}]
{\color{incolor}In [{\color{incolor}5}]:} \PY{n}{t} \PY{o}{=} \PY{n}{np}\PY{o}{.}\PY{n}{linspace}\PY{p}{(}\PY{l+m+mi}{0}\PY{p}{,}\PY{l+m+mi}{1}\PY{p}{,}\PY{l+m+mi}{500}\PY{p}{)}  \PY{c+c1}{\PYZsh{} 500 Zeitschritte von 0 bis 10s}
        \PY{n}{g} \PY{o}{=} \PY{l+m+mf}{9.81}
        \PY{n}{l} \PY{o}{=} \PY{l+m+mf}{0.01}
        
        \PY{k}{def} \PY{n+nf}{nlswing}\PY{p}{(}\PY{n}{x}\PY{p}{,}\PY{n}{t}\PY{p}{)}\PY{p}{:}
            \PY{k}{return}\PY{p}{[} \PY{n}{x}\PY{p}{[}\PY{l+m+mi}{1}\PY{p}{]}\PY{p}{,} \PY{o}{\PYZhy{}}\PY{n}{g}\PY{o}{/}\PY{n}{l}\PY{o}{*}\PY{n}{math}\PY{o}{.}\PY{n}{sin}\PY{p}{(}\PY{n}{x}\PY{p}{[}\PY{l+m+mi}{0}\PY{p}{]}\PY{p}{)} \PY{p}{]}
\end{Verbatim}


    Durch verschiedene Anfangsauslenkungen \(\theta_0\) und die
Anfangsgeschwindigkeiten \(v_0=0\) wird im folgenden das Verhalten
analysiert.

    \begin{Verbatim}[commandchars=\\\{\}]
{\color{incolor}In [{\color{incolor}6}]:} \PY{n}{theta0} \PY{o}{=} \PY{p}{[}\PY{l+m+mi}{1}\PY{p}{,} \PY{l+m+mi}{0}\PY{p}{]}     
        \PY{n}{theta} \PY{o}{=} \PY{n}{odeint}\PY{p}{(}\PY{n}{func}\PY{o}{=}\PY{n}{nlswing}\PY{p}{,} \PY{n}{y0}\PY{o}{=}\PY{n}{theta0}\PY{p}{,} \PY{n}{t}\PY{o}{=}\PY{n}{t}\PY{p}{)}
        \PY{n}{plt}\PY{o}{.}\PY{n}{plot}\PY{p}{(}\PY{n}{t}\PY{p}{,} \PY{n}{theta}\PY{p}{[}\PY{p}{:}\PY{p}{,}\PY{l+m+mi}{0}\PY{p}{]}\PY{p}{,} \PY{n}{label}\PY{o}{=}\PY{l+s+s2}{\PYZdq{}}\PY{l+s+s2}{theta=1, v=0}\PY{l+s+s2}{\PYZdq{}}\PY{p}{)}
        
        \PY{n}{theta0} \PY{o}{=} \PY{p}{[}\PY{l+m+mi}{2}\PY{p}{,} \PY{l+m+mi}{0}\PY{p}{]}        
        \PY{n}{theta} \PY{o}{=} \PY{n}{odeint}\PY{p}{(}\PY{n}{func}\PY{o}{=}\PY{n}{nlswing}\PY{p}{,} \PY{n}{y0}\PY{o}{=}\PY{n}{theta0}\PY{p}{,} \PY{n}{t}\PY{o}{=}\PY{n}{t}\PY{p}{)}
        \PY{n}{plt}\PY{o}{.}\PY{n}{plot}\PY{p}{(}\PY{n}{t}\PY{p}{,} \PY{n}{theta}\PY{p}{[}\PY{p}{:}\PY{p}{,}\PY{l+m+mi}{0}\PY{p}{]}\PY{p}{,} \PY{n}{label}\PY{o}{=}\PY{l+s+s2}{\PYZdq{}}\PY{l+s+s2}{theta=2, v=0}\PY{l+s+s2}{\PYZdq{}}\PY{p}{)}
        
        \PY{n}{theta0} \PY{o}{=} \PY{p}{[}\PY{l+m+mi}{3}\PY{p}{,} \PY{l+m+mi}{0}\PY{p}{]}        
        \PY{n}{theta} \PY{o}{=} \PY{n}{odeint}\PY{p}{(}\PY{n}{func}\PY{o}{=}\PY{n}{nlswing}\PY{p}{,} \PY{n}{y0}\PY{o}{=}\PY{n}{theta0}\PY{p}{,} \PY{n}{t}\PY{o}{=}\PY{n}{t}\PY{p}{)}
        \PY{n}{plt}\PY{o}{.}\PY{n}{plot}\PY{p}{(}\PY{n}{t}\PY{p}{,} \PY{n}{theta}\PY{p}{[}\PY{p}{:}\PY{p}{,}\PY{l+m+mi}{0}\PY{p}{]}\PY{p}{,} \PY{n}{label}\PY{o}{=}\PY{l+s+s2}{\PYZdq{}}\PY{l+s+s2}{theta=3, v=0}\PY{l+s+s2}{\PYZdq{}}\PY{p}{)}
        
        \PY{c+c1}{\PYZsh{} Plot Beschriftung}
        \PY{n}{plt}\PY{o}{.}\PY{n}{title}\PY{p}{(}\PY{l+s+s1}{\PYZsq{}}\PY{l+s+s1}{Weg/Zeit\PYZhy{}Diagramm}\PY{l+s+s1}{\PYZsq{}}\PY{p}{)}
        \PY{n}{plt}\PY{o}{.}\PY{n}{xlabel}\PY{p}{(}\PY{l+s+s1}{\PYZsq{}}\PY{l+s+s1}{Zeit (t)}\PY{l+s+s1}{\PYZsq{}}\PY{p}{)}
        \PY{n}{plt}\PY{o}{.}\PY{n}{ylabel}\PY{p}{(}\PY{l+s+s1}{\PYZsq{}}\PY{l+s+s1}{Weg (Theta)}\PY{l+s+s1}{\PYZsq{}}\PY{p}{)}
        \PY{n}{plt}\PY{o}{.}\PY{n}{legend}\PY{p}{(}\PY{p}{)}
\end{Verbatim}


\begin{Verbatim}[commandchars=\\\{\}]
{\color{outcolor}Out[{\color{outcolor}6}]:} <matplotlib.legend.Legend at 0x1157c0128>
\end{Verbatim}
            
    \begin{center}
    \adjustimage{max size={0.9\linewidth}{0.9\paperheight}}{output_14_1.png}
    \end{center}
    { \hspace*{\fill} \\}
    
    Im obigen Beispiel ist ersichtlich, dass mit steigender
Anfangsauslenkungen natürlich eine längere Schwingdauer in einer
Richtung und auch eine entsprechende Auslenkung erreicht wird. Da in
diesem Modell keine Reibung berücksichtigt wird, bleibt die Schwingung
dauerhaft erhalten.

Im nächsten Beispiel werden Grenzfälle analysiert.

    \begin{Verbatim}[commandchars=\\\{\}]
{\color{incolor}In [{\color{incolor}7}]:} \PY{n}{theta0} \PY{o}{=} \PY{p}{[}\PY{l+m+mi}{4}\PY{p}{,} \PY{l+m+mi}{0}\PY{p}{]}        
        \PY{n}{theta} \PY{o}{=} \PY{n}{odeint}\PY{p}{(}\PY{n}{func}\PY{o}{=}\PY{n}{nlswing}\PY{p}{,} \PY{n}{y0}\PY{o}{=}\PY{n}{theta0}\PY{p}{,} \PY{n}{t}\PY{o}{=}\PY{n}{t}\PY{p}{)}
        \PY{n}{plt}\PY{o}{.}\PY{n}{plot}\PY{p}{(}\PY{n}{t}\PY{p}{,} \PY{p}{[}\PY{n}{math}\PY{o}{.}\PY{n}{fmod}\PY{p}{(}\PY{n}{t}\PY{p}{,} \PY{l+m+mi}{2}\PY{o}{*}\PY{n}{math}\PY{o}{.}\PY{n}{pi}\PY{p}{)} \PY{k}{for} \PY{n}{t} \PY{o+ow}{in} \PY{n}{theta}\PY{p}{[}\PY{p}{:}\PY{p}{,}\PY{l+m+mi}{0}\PY{p}{]}\PY{p}{]}\PY{p}{,} \PY{n}{label}\PY{o}{=}\PY{l+s+s2}{\PYZdq{}}\PY{l+s+s2}{theta=4, v=0}\PY{l+s+s2}{\PYZdq{}}\PY{p}{)}
        
        \PY{n}{theta0} \PY{o}{=} \PY{p}{[}\PY{n}{math}\PY{o}{.}\PY{n}{pi}\PY{p}{,} \PY{l+m+mi}{0}\PY{p}{]}        
        \PY{n}{theta} \PY{o}{=} \PY{n}{odeint}\PY{p}{(}\PY{n}{func}\PY{o}{=}\PY{n}{nlswing}\PY{p}{,} \PY{n}{y0}\PY{o}{=}\PY{n}{theta0}\PY{p}{,} \PY{n}{t}\PY{o}{=}\PY{n}{t}\PY{p}{)}
        \PY{n}{plt}\PY{o}{.}\PY{n}{plot}\PY{p}{(}\PY{n}{t}\PY{p}{,}\PY{n}{theta}\PY{p}{[}\PY{p}{:}\PY{p}{,}\PY{l+m+mi}{0}\PY{p}{]}\PY{p}{,} \PY{n}{label}\PY{o}{=}\PY{l+s+s2}{\PYZdq{}}\PY{l+s+s2}{theta=pi, v=0}\PY{l+s+s2}{\PYZdq{}}\PY{p}{)}
        
        \PY{n}{theta0} \PY{o}{=} \PY{p}{[}\PY{l+m+mi}{0}\PY{p}{,} \PY{l+m+mi}{0}\PY{p}{]}        
        \PY{n}{theta} \PY{o}{=} \PY{n}{odeint}\PY{p}{(}\PY{n}{func}\PY{o}{=}\PY{n}{nlswing}\PY{p}{,} \PY{n}{y0}\PY{o}{=}\PY{n}{theta0}\PY{p}{,} \PY{n}{t}\PY{o}{=}\PY{n}{t}\PY{p}{)}
        \PY{n}{plt}\PY{o}{.}\PY{n}{plot}\PY{p}{(}\PY{n}{t}\PY{p}{,}\PY{n}{theta}\PY{p}{[}\PY{p}{:}\PY{p}{,}\PY{l+m+mi}{0}\PY{p}{]}\PY{p}{,} \PY{n}{label}\PY{o}{=}\PY{l+s+s2}{\PYZdq{}}\PY{l+s+s2}{theta=0, v=0}\PY{l+s+s2}{\PYZdq{}}\PY{p}{)}
        
        \PY{c+c1}{\PYZsh{} Plot Beschriftung}
        \PY{n}{plt}\PY{o}{.}\PY{n}{title}\PY{p}{(}\PY{l+s+s1}{\PYZsq{}}\PY{l+s+s1}{Weg/Zeit\PYZhy{}Diagramm}\PY{l+s+s1}{\PYZsq{}}\PY{p}{)}
        \PY{n}{plt}\PY{o}{.}\PY{n}{xlabel}\PY{p}{(}\PY{l+s+s1}{\PYZsq{}}\PY{l+s+s1}{Zeit (t)}\PY{l+s+s1}{\PYZsq{}}\PY{p}{)}
        \PY{n}{plt}\PY{o}{.}\PY{n}{ylabel}\PY{p}{(}\PY{l+s+s1}{\PYZsq{}}\PY{l+s+s1}{Weg (Theta)}\PY{l+s+s1}{\PYZsq{}}\PY{p}{)}
        \PY{n}{plt}\PY{o}{.}\PY{n}{legend}\PY{p}{(}\PY{p}{)}
\end{Verbatim}


\begin{Verbatim}[commandchars=\\\{\}]
{\color{outcolor}Out[{\color{outcolor}7}]:} <matplotlib.legend.Legend at 0x115836668>
\end{Verbatim}
            
    \begin{center}
    \adjustimage{max size={0.9\linewidth}{0.9\paperheight}}{output_16_1.png}
    \end{center}
    { \hspace*{\fill} \\}
    
    Im obigen Beispiel sind für die Anfangsauslenkung die Ruhepositionen
beim Winkel \(\theta=0\) und beim senkrechten Stand mit \(\theta=\pi\)
dargestellt. Hier bleibt die Position erhalten, da sich die Kräfte in
beiden Positionen ausgleichen. In der Praxis wäre dies nur schwer zu
erreichen, da durch äußere Einflüsse, gerade beim senkrachtend Stand
nach Oben, die Ruhe schnell gekippt werden würde.

Zusätzlich ist noch ein Fall dargestellt, bei dem die Anfangsauslenkung
über die höchste Position \(\theta=\pi\) gezogen wurde. Dies verhält
sich jedoch ähnlich zu dem Fall, dass das Pendel an die gleiche Position
bloß einfach in die entgegengesetzte Richtung hochgezogen würde. Durch
einfaches verschieben der Kurven würde die normale Schwingung
dargestellt werden.

Im nächsten Beispiel sind verschiedene Anfangsgeschwindigkeiten
dargestellt:

    \begin{Verbatim}[commandchars=\\\{\}]
{\color{incolor}In [{\color{incolor}8}]:} \PY{n}{theta0} \PY{o}{=} \PY{p}{[}\PY{l+m+mi}{1}\PY{p}{,} \PY{l+m+mi}{0}\PY{p}{]}     
        \PY{n}{theta} \PY{o}{=} \PY{n}{odeint}\PY{p}{(}\PY{n}{func}\PY{o}{=}\PY{n}{nlswing}\PY{p}{,} \PY{n}{y0}\PY{o}{=}\PY{n}{theta0}\PY{p}{,} \PY{n}{t}\PY{o}{=}\PY{n}{t}\PY{p}{)}
        \PY{n}{plt}\PY{o}{.}\PY{n}{plot}\PY{p}{(}\PY{n}{t}\PY{p}{,} \PY{n}{theta}\PY{p}{[}\PY{p}{:}\PY{p}{,}\PY{l+m+mi}{0}\PY{p}{]}\PY{p}{,} \PY{n}{label}\PY{o}{=}\PY{l+s+s2}{\PYZdq{}}\PY{l+s+s2}{theta=1, v=0}\PY{l+s+s2}{\PYZdq{}}\PY{p}{)}
        
        \PY{n}{theta0} \PY{o}{=} \PY{p}{[}\PY{l+m+mi}{1}\PY{p}{,} \PY{l+m+mi}{20}\PY{p}{]}        
        \PY{n}{theta} \PY{o}{=} \PY{n}{odeint}\PY{p}{(}\PY{n}{func}\PY{o}{=}\PY{n}{nlswing}\PY{p}{,} \PY{n}{y0}\PY{o}{=}\PY{n}{theta0}\PY{p}{,} \PY{n}{t}\PY{o}{=}\PY{n}{t}\PY{p}{)}
        \PY{n}{plt}\PY{o}{.}\PY{n}{plot}\PY{p}{(}\PY{n}{t}\PY{p}{,} \PY{n}{theta}\PY{p}{[}\PY{p}{:}\PY{p}{,}\PY{l+m+mi}{0}\PY{p}{]}\PY{p}{,} \PY{n}{label}\PY{o}{=}\PY{l+s+s2}{\PYZdq{}}\PY{l+s+s2}{theta=1, v=20}\PY{l+s+s2}{\PYZdq{}}\PY{p}{)}
        
        \PY{n}{theta0} \PY{o}{=} \PY{p}{[}\PY{l+m+mi}{1}\PY{p}{,} \PY{l+m+mi}{30}\PY{p}{]}        
        \PY{n}{theta} \PY{o}{=} \PY{n}{odeint}\PY{p}{(}\PY{n}{func}\PY{o}{=}\PY{n}{nlswing}\PY{p}{,} \PY{n}{y0}\PY{o}{=}\PY{n}{theta0}\PY{p}{,} \PY{n}{t}\PY{o}{=}\PY{n}{t}\PY{p}{)}
        \PY{n}{plt}\PY{o}{.}\PY{n}{plot}\PY{p}{(}\PY{n}{t}\PY{p}{,} \PY{n}{theta}\PY{p}{[}\PY{p}{:}\PY{p}{,}\PY{l+m+mi}{0}\PY{p}{]}\PY{p}{,} \PY{n}{label}\PY{o}{=}\PY{l+s+s2}{\PYZdq{}}\PY{l+s+s2}{theta=1, v=30}\PY{l+s+s2}{\PYZdq{}}\PY{p}{)}
        
        \PY{c+c1}{\PYZsh{} Plot Beschriftung}
        \PY{n}{plt}\PY{o}{.}\PY{n}{title}\PY{p}{(}\PY{l+s+s1}{\PYZsq{}}\PY{l+s+s1}{Weg/Zeit\PYZhy{}Diagramm}\PY{l+s+s1}{\PYZsq{}}\PY{p}{)}
        \PY{n}{plt}\PY{o}{.}\PY{n}{xlabel}\PY{p}{(}\PY{l+s+s1}{\PYZsq{}}\PY{l+s+s1}{Zeit (t)}\PY{l+s+s1}{\PYZsq{}}\PY{p}{)}
        \PY{n}{plt}\PY{o}{.}\PY{n}{ylabel}\PY{p}{(}\PY{l+s+s1}{\PYZsq{}}\PY{l+s+s1}{Weg (Theta)}\PY{l+s+s1}{\PYZsq{}}\PY{p}{)}
        \PY{n}{plt}\PY{o}{.}\PY{n}{legend}\PY{p}{(}\PY{p}{)}
\end{Verbatim}


\begin{Verbatim}[commandchars=\\\{\}]
{\color{outcolor}Out[{\color{outcolor}8}]:} <matplotlib.legend.Legend at 0x11597a828>
\end{Verbatim}
            
    \begin{center}
    \adjustimage{max size={0.9\linewidth}{0.9\paperheight}}{output_18_1.png}
    \end{center}
    { \hspace*{\fill} \\}
    
    Da die Anfangsgeschwindigkeit nicht 0 ist, bewegt sich das Pendel
anfangs noch nach oben. Je höher \(v_0\) ist, desto weiter bewegt sich
das Pendel nach oben. Nachdem der Höhepunkt erreicht wurde, verhält sich
das Pendel so, als hätte man es am Höhepunkt ohne Anfangsgeschwindigkeit
losgelassen.

Im nächsten Plot sind wieder interessante Verhalten für besondere
Anfangsgeschwindigkeiten zu sehen:

    \begin{Verbatim}[commandchars=\\\{\}]
{\color{incolor}In [{\color{incolor}9}]:} \PY{n}{theta0} \PY{o}{=} \PY{p}{[}\PY{l+m+mi}{1}\PY{p}{,} \PY{l+m+mi}{50}\PY{p}{]}        
        \PY{n}{theta} \PY{o}{=} \PY{n}{odeint}\PY{p}{(}\PY{n}{func}\PY{o}{=}\PY{n}{nlswing}\PY{p}{,} \PY{n}{y0}\PY{o}{=}\PY{n}{theta0}\PY{p}{,} \PY{n}{t}\PY{o}{=}\PY{n}{t}\PY{p}{)}
        \PY{n}{plt}\PY{o}{.}\PY{n}{plot}\PY{p}{(}\PY{n}{t}\PY{p}{,} \PY{n}{theta}\PY{p}{[}\PY{p}{:}\PY{p}{,}\PY{l+m+mi}{0}\PY{p}{]}\PY{p}{,} \PY{n}{label}\PY{o}{=}\PY{l+s+s2}{\PYZdq{}}\PY{l+s+s2}{theta=1, v=50}\PY{l+s+s2}{\PYZdq{}}\PY{p}{)}
        
        \PY{n}{theta0} \PY{o}{=} \PY{p}{[}\PY{l+m+mi}{1}\PY{p}{,} \PY{l+m+mi}{55}\PY{p}{]}        
        \PY{n}{theta} \PY{o}{=} \PY{n}{odeint}\PY{p}{(}\PY{n}{func}\PY{o}{=}\PY{n}{nlswing}\PY{p}{,} \PY{n}{y0}\PY{o}{=}\PY{n}{theta0}\PY{p}{,} \PY{n}{t}\PY{o}{=}\PY{n}{t}\PY{p}{)}
        \PY{n}{plt}\PY{o}{.}\PY{n}{plot}\PY{p}{(}\PY{n}{t}\PY{p}{,} \PY{n}{theta}\PY{p}{[}\PY{p}{:}\PY{p}{,}\PY{l+m+mi}{0}\PY{p}{]}\PY{p}{,} \PY{n}{label}\PY{o}{=}\PY{l+s+s2}{\PYZdq{}}\PY{l+s+s2}{theta=1, v=55}\PY{l+s+s2}{\PYZdq{}}\PY{p}{)}
        
        \PY{n}{theta0} \PY{o}{=} \PY{p}{[}\PY{l+m+mi}{1}\PY{p}{,} \PY{o}{\PYZhy{}}\PY{l+m+mi}{55}\PY{p}{]}        
        \PY{n}{theta} \PY{o}{=} \PY{n}{odeint}\PY{p}{(}\PY{n}{func}\PY{o}{=}\PY{n}{nlswing}\PY{p}{,} \PY{n}{y0}\PY{o}{=}\PY{n}{theta0}\PY{p}{,} \PY{n}{t}\PY{o}{=}\PY{n}{t}\PY{p}{)}
        \PY{n}{plt}\PY{o}{.}\PY{n}{plot}\PY{p}{(}\PY{n}{t}\PY{p}{,} \PY{n}{theta}\PY{p}{[}\PY{p}{:}\PY{p}{,}\PY{l+m+mi}{0}\PY{p}{]}\PY{p}{,} \PY{n}{label}\PY{o}{=}\PY{l+s+s2}{\PYZdq{}}\PY{l+s+s2}{theta=1, v=\PYZhy{}55}\PY{l+s+s2}{\PYZdq{}}\PY{p}{)}
        
        \PY{c+c1}{\PYZsh{} Plot Beschriftung}
        \PY{n}{plt}\PY{o}{.}\PY{n}{title}\PY{p}{(}\PY{l+s+s1}{\PYZsq{}}\PY{l+s+s1}{Weg/Zeit\PYZhy{}Diagramm}\PY{l+s+s1}{\PYZsq{}}\PY{p}{)}
        \PY{n}{plt}\PY{o}{.}\PY{n}{xlabel}\PY{p}{(}\PY{l+s+s1}{\PYZsq{}}\PY{l+s+s1}{Zeit (t)}\PY{l+s+s1}{\PYZsq{}}\PY{p}{)}
        \PY{n}{plt}\PY{o}{.}\PY{n}{ylabel}\PY{p}{(}\PY{l+s+s1}{\PYZsq{}}\PY{l+s+s1}{Weg (Theta)}\PY{l+s+s1}{\PYZsq{}}\PY{p}{)}
        \PY{n}{plt}\PY{o}{.}\PY{n}{legend}\PY{p}{(}\PY{p}{)}
\end{Verbatim}


\begin{Verbatim}[commandchars=\\\{\}]
{\color{outcolor}Out[{\color{outcolor}9}]:} <matplotlib.legend.Legend at 0x115a58be0>
\end{Verbatim}
            
    \begin{center}
    \adjustimage{max size={0.9\linewidth}{0.9\paperheight}}{output_20_1.png}
    \end{center}
    { \hspace*{\fill} \\}
    
    Bis zu einer Anfangsgeschwindigkeit von etwa 50m/s kommt das Pendel noch
vor dem senkrechtend Stand mit \(\theta=\pi\) zum stehen und kommt in
den Schwingfall. Wenn die Geschwindigkeit höher gewählt wird, wird die
höchste Position überlaufen und das Pendel dreht sich in einer Richtung
im Kreis. Dies ist daran zu erkennen, dass der Winkel nicht um den
Null-Punkt schwingt. Der Dargestellte Winkel ist somit die Anzahl an
Umdrehungen, die das Pendel schon absolviert hat.

Im folgenden ist noch der Phasenraum für verschiedene bereits oben
dargestellte Anfangspositionen geplottet.

    \begin{Verbatim}[commandchars=\\\{\}]
{\color{incolor}In [{\color{incolor}10}]:} \PY{n}{theta0} \PY{o}{=} \PY{p}{[}\PY{l+m+mi}{1}\PY{p}{,} \PY{l+m+mi}{0}\PY{p}{]}     
         \PY{n}{theta} \PY{o}{=} \PY{n}{odeint}\PY{p}{(}\PY{n}{func}\PY{o}{=}\PY{n}{nlswing}\PY{p}{,} \PY{n}{y0}\PY{o}{=}\PY{n}{theta0}\PY{p}{,} \PY{n}{t}\PY{o}{=}\PY{n}{t}\PY{p}{)}
         \PY{n}{plt}\PY{o}{.}\PY{n}{plot}\PY{p}{(}\PY{p}{[}\PY{n}{math}\PY{o}{.}\PY{n}{fmod}\PY{p}{(}\PY{n}{t}\PY{p}{,}\PY{l+m+mi}{2}\PY{o}{*}\PY{n}{math}\PY{o}{.}\PY{n}{pi}\PY{p}{)} \PY{k}{for} \PY{n}{t} \PY{o+ow}{in} \PY{n}{theta}\PY{p}{[}\PY{p}{:}\PY{p}{,}\PY{l+m+mi}{0}\PY{p}{]}\PY{p}{]}\PY{p}{,} \PY{n}{theta}\PY{p}{[}\PY{p}{:}\PY{p}{,}\PY{l+m+mi}{1}\PY{p}{]}\PY{p}{,} \PY{n}{label}\PY{o}{=}\PY{l+s+s2}{\PYZdq{}}\PY{l+s+s2}{theta=1, v=0}\PY{l+s+s2}{\PYZdq{}}\PY{p}{)}
         
         \PY{n}{theta0} \PY{o}{=} \PY{p}{[}\PY{l+m+mi}{3}\PY{p}{,} \PY{l+m+mi}{0}\PY{p}{]}     
         \PY{n}{theta} \PY{o}{=} \PY{n}{odeint}\PY{p}{(}\PY{n}{func}\PY{o}{=}\PY{n}{nlswing}\PY{p}{,} \PY{n}{y0}\PY{o}{=}\PY{n}{theta0}\PY{p}{,} \PY{n}{t}\PY{o}{=}\PY{n}{t}\PY{p}{)}
         \PY{n}{plt}\PY{o}{.}\PY{n}{plot}\PY{p}{(}\PY{p}{[}\PY{n}{math}\PY{o}{.}\PY{n}{fmod}\PY{p}{(}\PY{n}{t}\PY{p}{,}\PY{l+m+mi}{2}\PY{o}{*}\PY{n}{math}\PY{o}{.}\PY{n}{pi}\PY{p}{)} \PY{k}{for} \PY{n}{t} \PY{o+ow}{in} \PY{n}{theta}\PY{p}{[}\PY{p}{:}\PY{p}{,}\PY{l+m+mi}{0}\PY{p}{]}\PY{p}{]}\PY{p}{,} \PY{n}{theta}\PY{p}{[}\PY{p}{:}\PY{p}{,}\PY{l+m+mi}{1}\PY{p}{]}\PY{p}{,} \PY{n}{label}\PY{o}{=}\PY{l+s+s2}{\PYZdq{}}\PY{l+s+s2}{theta=3, v=0}\PY{l+s+s2}{\PYZdq{}}\PY{p}{)}
         
         \PY{n}{theta0} \PY{o}{=} \PY{p}{[}\PY{l+m+mi}{4}\PY{p}{,} \PY{l+m+mi}{0}\PY{p}{]}     
         \PY{n}{theta} \PY{o}{=} \PY{n}{odeint}\PY{p}{(}\PY{n}{func}\PY{o}{=}\PY{n}{nlswing}\PY{p}{,} \PY{n}{y0}\PY{o}{=}\PY{n}{theta0}\PY{p}{,} \PY{n}{t}\PY{o}{=}\PY{n}{t}\PY{p}{)}
         \PY{n}{plt}\PY{o}{.}\PY{n}{plot}\PY{p}{(}\PY{p}{[}\PY{n}{math}\PY{o}{.}\PY{n}{fmod}\PY{p}{(}\PY{n}{t}\PY{p}{,}\PY{l+m+mi}{2}\PY{o}{*}\PY{n}{math}\PY{o}{.}\PY{n}{pi}\PY{p}{)} \PY{k}{for} \PY{n}{t} \PY{o+ow}{in} \PY{n}{theta}\PY{p}{[}\PY{p}{:}\PY{p}{,}\PY{l+m+mi}{0}\PY{p}{]}\PY{p}{]}\PY{p}{,} \PY{n}{theta}\PY{p}{[}\PY{p}{:}\PY{p}{,}\PY{l+m+mi}{1}\PY{p}{]}\PY{p}{,} \PY{n}{label}\PY{o}{=}\PY{l+s+s2}{\PYZdq{}}\PY{l+s+s2}{theta=4, v=0}\PY{l+s+s2}{\PYZdq{}}\PY{p}{)}
         
         \PY{n}{theta0} \PY{o}{=} \PY{p}{[}\PY{l+m+mi}{1}\PY{p}{,} \PY{l+m+mi}{55}\PY{p}{]}     
         \PY{n}{theta} \PY{o}{=} \PY{n}{odeint}\PY{p}{(}\PY{n}{func}\PY{o}{=}\PY{n}{nlswing}\PY{p}{,} \PY{n}{y0}\PY{o}{=}\PY{n}{theta0}\PY{p}{,} \PY{n}{t}\PY{o}{=}\PY{n}{t}\PY{p}{)}
         \PY{n}{plt}\PY{o}{.}\PY{n}{plot}\PY{p}{(}\PY{p}{[}\PY{n}{math}\PY{o}{.}\PY{n}{fmod}\PY{p}{(}\PY{n}{t}\PY{p}{,}\PY{l+m+mi}{2}\PY{o}{*}\PY{n}{math}\PY{o}{.}\PY{n}{pi}\PY{p}{)} \PY{k}{for} \PY{n}{t} \PY{o+ow}{in} \PY{n}{theta}\PY{p}{[}\PY{p}{:}\PY{p}{,}\PY{l+m+mi}{0}\PY{p}{]}\PY{p}{]}\PY{p}{,} \PY{n}{theta}\PY{p}{[}\PY{p}{:}\PY{p}{,}\PY{l+m+mi}{1}\PY{p}{]}\PY{p}{,} \PY{n}{label}\PY{o}{=}\PY{l+s+s2}{\PYZdq{}}\PY{l+s+s2}{theta=1, v=55}\PY{l+s+s2}{\PYZdq{}}\PY{p}{)}
         
         \PY{c+c1}{\PYZsh{} Plot Beschriftung}
         \PY{n}{plt}\PY{o}{.}\PY{n}{title}\PY{p}{(}\PY{l+s+s1}{\PYZsq{}}\PY{l+s+s1}{Phasenraum}\PY{l+s+s1}{\PYZsq{}}\PY{p}{)}
         \PY{n}{plt}\PY{o}{.}\PY{n}{xlabel}\PY{p}{(}\PY{l+s+s1}{\PYZsq{}}\PY{l+s+s1}{Theta}\PY{l+s+s1}{\PYZsq{}}\PY{p}{)}
         \PY{n}{plt}\PY{o}{.}\PY{n}{ylabel}\PY{p}{(}\PY{l+s+s1}{\PYZsq{}}\PY{l+s+s1}{Winkelgeschwindigkeit (v)}\PY{l+s+s1}{\PYZsq{}}\PY{p}{)}
         \PY{n}{plt}\PY{o}{.}\PY{n}{legend}\PY{p}{(}\PY{p}{)}
\end{Verbatim}


\begin{Verbatim}[commandchars=\\\{\}]
{\color{outcolor}Out[{\color{outcolor}10}]:} <matplotlib.legend.Legend at 0x115b4b3c8>
\end{Verbatim}
            
    \begin{center}
    \adjustimage{max size={0.9\linewidth}{0.9\paperheight}}{output_22_1.png}
    \end{center}
    { \hspace*{\fill} \\}
    
    \subsection{Aufgabe 3: Erzwungene
Schwingung}\label{aufgabe-3-erzwungene-schwingung}

Zunächst werden wird wieder das Gleichungssystem zur Lösung des DGL
aufgestellt.

\[
y_1 = x \\
y_2 = \dot{x} \\
\dot{y_1} = y_2 = \dot{x} \\
\dot{y_2} = \ddot{x} = \frac{1}{m}(F_0 \cdot cos{\omega t} -D\dot{x}-cx )
\]

    Der Zustand zum Zeitpunkt t = 0 wird festgelegt als:

\begin{align}
\vec{z_0}= \left( \begin{array}{c}x_0\\v_0\end{array} \right) = \left( \begin{array}{c}0\\0\end{array} \right)
\end{align}

    \begin{Verbatim}[commandchars=\\\{\}]
{\color{incolor}In [{\color{incolor}343}]:} \PY{n}{t} \PY{o}{=} \PY{n}{np}\PY{o}{.}\PY{n}{linspace}\PY{p}{(}\PY{l+m+mi}{0}\PY{p}{,}\PY{l+m+mi}{100}\PY{p}{,} \PY{n}{num}\PY{o}{=}\PY{l+m+mi}{1000}\PY{p}{)}
          
          \PY{c+c1}{\PYZsh{}Definierte Randbedingungen}
          \PY{n}{omega\PYZus{}0} \PY{o}{=} \PY{l+m+mi}{5}
          \PY{n}{m} \PY{o}{=} \PY{l+m+mi}{3}
          \PY{n}{D} \PY{o}{=} \PY{l+m+mi}{1}\PY{o}{/}\PY{n}{omega\PYZus{}0}
          \PY{n}{F0} \PY{o}{=} \PY{l+m+mi}{30}
          \PY{n}{c} \PY{o}{=} \PY{l+m+mf}{0.2}
          
          \PY{k}{def} \PY{n+nf}{erzSwing}\PY{p}{(}\PY{n}{x}\PY{p}{,}\PY{n}{t}\PY{p}{,}\PY{n}{om}\PY{p}{)}\PY{p}{:}
              \PY{k}{return}\PY{p}{[} \PY{n}{x}\PY{p}{[}\PY{l+m+mi}{1}\PY{p}{]}\PY{p}{,} \PY{p}{(}\PY{n}{F0}\PY{o}{*}\PY{n}{math}\PY{o}{.}\PY{n}{cos}\PY{p}{(}\PY{n}{om}\PY{o}{*}\PY{n}{t}\PY{p}{)}\PY{o}{\PYZhy{}}\PY{n}{D}\PY{o}{*}\PY{n}{x}\PY{p}{[}\PY{l+m+mi}{0}\PY{p}{]}\PY{o}{\PYZhy{}}\PY{n}{c}\PY{o}{*}\PY{n}{x}\PY{p}{[}\PY{l+m+mi}{1}\PY{p}{]}\PY{p}{)}\PY{o}{/}\PY{n}{m} \PY{p}{]}
          
          
          \PY{n}{start} \PY{o}{=} \PY{p}{[}\PY{l+m+mi}{0}\PY{p}{,} \PY{l+m+mi}{0}\PY{p}{]}   
          \PY{n}{omega} \PY{o}{=} \PY{l+m+mi}{5}
          \PY{n}{theta} \PY{o}{=} \PY{n}{odeint}\PY{p}{(}\PY{n}{func}\PY{o}{=}\PY{n}{erzSwing}\PY{p}{,} \PY{n}{y0}\PY{o}{=}\PY{n}{start}\PY{p}{,} \PY{n}{t}\PY{o}{=}\PY{n}{t}\PY{p}{,} \PY{n}{args}\PY{o}{=}\PY{p}{(}\PY{n}{omega}\PY{p}{,}\PY{p}{)}\PY{p}{)}
          \PY{n}{plt}\PY{o}{.}\PY{n}{plot}\PY{p}{(}\PY{n}{t}\PY{p}{,} \PY{n}{theta}\PY{p}{[}\PY{p}{:}\PY{p}{,}\PY{l+m+mi}{0}\PY{p}{]}\PY{p}{)}
          
          \PY{n}{plt}\PY{o}{.}\PY{n}{title}\PY{p}{(}\PY{l+s+s1}{\PYZsq{}}\PY{l+s+s1}{Zeit\PYZhy{}Weg\PYZhy{}Diagramm}\PY{l+s+s1}{\PYZsq{}}\PY{p}{)}
          \PY{n}{plt}\PY{o}{.}\PY{n}{xlabel}\PY{p}{(}\PY{l+s+s1}{\PYZsq{}}\PY{l+s+s1}{Zeit (t)}\PY{l+s+s1}{\PYZsq{}}\PY{p}{)}
          \PY{n}{plt}\PY{o}{.}\PY{n}{ylabel}\PY{p}{(}\PY{l+s+s1}{\PYZsq{}}\PY{l+s+s1}{Amplitude}\PY{l+s+s1}{\PYZsq{}}\PY{p}{)}
\end{Verbatim}


\begin{Verbatim}[commandchars=\\\{\}]
{\color{outcolor}Out[{\color{outcolor}343}]:} Text(0,0.5,'Amplitude')
\end{Verbatim}
            
    \begin{center}
    \adjustimage{max size={0.9\linewidth}{0.9\paperheight}}{output_25_1.png}
    \end{center}
    { \hspace*{\fill} \\}
    
    Im obigen Bild ist die Einschwingphase gut zu erkennen.

Im folgenden wird die Resonanzkurve berechnet.

    \begin{Verbatim}[commandchars=\\\{\}]
{\color{incolor}In [{\color{incolor}342}]:} \PY{k}{def} \PY{n+nf}{resAnalytisch}\PY{p}{(}\PY{n}{omega}\PY{p}{)}\PY{p}{:}
              \PY{n}{n} \PY{o}{=} \PY{n}{omega}\PY{o}{/}\PY{n}{omega\PYZus{}0}
              \PY{k}{return} \PY{l+m+mi}{1}\PY{o}{/}\PY{n}{np}\PY{o}{.}\PY{n}{sqrt}\PY{p}{(}\PY{p}{(}\PY{l+m+mi}{1}\PY{o}{\PYZhy{}}\PY{n}{n}\PY{o}{*}\PY{o}{*}\PY{l+m+mi}{2}\PY{p}{)}\PY{o}{*}\PY{o}{*}\PY{l+m+mi}{2}\PY{o}{+}\PY{l+m+mi}{4}\PY{o}{*}\PY{p}{(}\PY{n}{D}\PY{o}{*}\PY{o}{*}\PY{l+m+mi}{2}\PY{p}{)}\PY{o}{*}\PY{p}{(}\PY{n}{n}\PY{o}{*}\PY{o}{*}\PY{l+m+mi}{2}\PY{p}{)}\PY{p}{)}
          
          \PY{k}{def} \PY{n+nf}{resNum}\PY{p}{(}\PY{n}{omega}\PY{p}{)}\PY{p}{:}
              \PY{n}{theta} \PY{o}{=} \PY{n}{odeint}\PY{p}{(}\PY{n}{func}\PY{o}{=}\PY{n}{erzSwing}\PY{p}{,} \PY{n}{y0}\PY{o}{=}\PY{n}{start}\PY{p}{,} \PY{n}{t}\PY{o}{=}\PY{n}{t}\PY{p}{,} \PY{n}{args}\PY{o}{=}\PY{p}{(}\PY{n}{omega}\PY{p}{,}\PY{p}{)}\PY{p}{)}
              \PY{k}{return} \PY{n+nb}{max}\PY{p}{(}\PY{p}{[}\PY{n}{x}\PY{p}{[}\PY{l+m+mi}{0}\PY{p}{]} \PY{k}{for} \PY{n}{x} \PY{o+ow}{in} \PY{n}{theta}\PY{p}{[}\PY{l+m+mi}{900}\PY{p}{:}\PY{l+m+mi}{1000}\PY{p}{]}\PY{p}{]}\PY{p}{)}\PY{o}{/}\PY{n}{F0}
              
              
          \PY{n}{omega\PYZus{}0} \PY{o}{=} \PY{l+m+mi}{6}
          \PY{n}{r} \PY{o}{=} \PY{n}{np}\PY{o}{.}\PY{n}{arange}\PY{p}{(} \PY{l+m+mf}{0.01}\PY{p}{,} \PY{l+m+mi}{10}\PY{p}{,} \PY{l+m+mf}{0.1}\PY{p}{)}
          \PY{n}{plt}\PY{o}{.}\PY{n}{plot}\PY{p}{(}\PY{n}{r}\PY{p}{,} \PY{n}{resAnalytisch}\PY{p}{(}\PY{n}{r}\PY{p}{)}\PY{p}{,}\PY{n}{label}\PY{o}{=}\PY{l+s+s1}{\PYZsq{}}\PY{l+s+s1}{analytisch}\PY{l+s+s1}{\PYZsq{}}\PY{p}{)}
          \PY{n}{plt}\PY{o}{.}\PY{n}{plot}\PY{p}{(}\PY{n}{r}\PY{p}{,} \PY{p}{[}\PY{n}{resNum}\PY{p}{(}\PY{n}{x}\PY{p}{)} \PY{k}{for} \PY{n}{x} \PY{o+ow}{in} \PY{n}{r}\PY{p}{]}\PY{p}{,}\PY{n}{label}\PY{o}{=}\PY{l+s+s1}{\PYZsq{}}\PY{l+s+s1}{numerisch}\PY{l+s+s1}{\PYZsq{}}\PY{p}{)}
          
          
          \PY{c+c1}{\PYZsh{}Plotbeschriftung}
          \PY{n}{plt}\PY{o}{.}\PY{n}{title}\PY{p}{(}\PY{l+s+s1}{\PYZsq{}}\PY{l+s+s1}{Amplitudengang}\PY{l+s+s1}{\PYZsq{}}\PY{p}{)}
          \PY{n}{plt}\PY{o}{.}\PY{n}{ylabel}\PY{p}{(}\PY{l+s+s1}{\PYZsq{}}\PY{l+s+s1}{Amplitude}\PY{l+s+s1}{\PYZsq{}}\PY{p}{)}
          \PY{n}{plt}\PY{o}{.}\PY{n}{xlabel}\PY{p}{(}\PY{l+s+s1}{\PYZsq{}}\PY{l+s+s1}{Kreisfrequenz}\PY{l+s+s1}{\PYZsq{}}\PY{p}{)}
          \PY{n}{plt}\PY{o}{.}\PY{n}{legend}\PY{p}{(}\PY{p}{)}
\end{Verbatim}


\begin{Verbatim}[commandchars=\\\{\}]
{\color{outcolor}Out[{\color{outcolor}342}]:} <matplotlib.legend.Legend at 0x155e639908>
\end{Verbatim}
            
    \begin{center}
    \adjustimage{max size={0.9\linewidth}{0.9\paperheight}}{output_27_1.png}
    \end{center}
    { \hspace*{\fill} \\}
    
    In einem nächsten Schritt wird ein kleinerer nicht linearer Teil
hinzugefügt. Alle anderen Werte bleiben identisch.

    \begin{Verbatim}[commandchars=\\\{\}]
{\color{incolor}In [{\color{incolor}13}]:} \PY{n}{z1}\PY{o}{=}\PY{p}{[}\PY{p}{[}\PY{l+m+mi}{0}\PY{p}{,}\PY{l+m+mi}{0}\PY{p}{]}\PY{p}{]}
         \PY{n}{z2}\PY{o}{=}\PY{p}{[}\PY{p}{[}\PY{l+m+mi}{0}\PY{p}{,}\PY{l+m+mi}{0}\PY{p}{]}\PY{p}{]}
         \PY{n}{z3}\PY{o}{=}\PY{p}{[}\PY{p}{[}\PY{l+m+mi}{0}\PY{p}{,}\PY{l+m+mi}{0}\PY{p}{]}\PY{p}{]}
         
         \PY{n}{m} \PY{o}{=} \PY{l+m+mi}{1}
         
         \PY{n}{dt} \PY{o}{=} \PY{l+m+mf}{0.00005}
         \PY{n}{D} \PY{o}{=} \PY{l+m+mi}{4} \PY{o}{*} \PY{n}{math}\PY{o}{.}\PY{n}{pi} \PY{o}{*}\PY{o}{*} \PY{l+m+mi}{2} \PY{o}{*} \PY{n}{m} 
         \PY{n}{D3}\PY{o}{=} \PY{l+m+mi}{4} \PY{o}{*} \PY{n}{math}\PY{o}{.}\PY{n}{pi} \PY{o}{*}\PY{o}{*} \PY{l+m+mi}{2} \PY{o}{*} \PY{n}{m} \PY{o}{/} \PY{l+m+mi}{20}
         \PY{n}{omega1} \PY{o}{=} \PY{l+m+mi}{1} \PY{o}{*} \PY{p}{(}\PY{l+m+mi}{2} \PY{o}{*} \PY{n}{math}\PY{o}{.}\PY{n}{pi}\PY{p}{)} \PY{o}{*} \PY{l+m+mf}{0.97}          \PY{c+c1}{\PYZsh{} (2 * math.pi) =\PYZgt{} Eine Schwingung pro Sekunde}
         \PY{n}{omega2} \PY{o}{=} \PY{l+m+mi}{1} \PY{o}{*} \PY{p}{(}\PY{l+m+mi}{2} \PY{o}{*} \PY{n}{math}\PY{o}{.}\PY{n}{pi}\PY{p}{)}           \PY{c+c1}{\PYZsh{} (2 * math.pi) =\PYZgt{} Eine Schwingung pro Sekunde}
         \PY{n}{F0} \PY{o}{=} \PY{l+m+mi}{20}
         \PY{n}{c} \PY{o}{=} \PY{l+m+mf}{0.1}
         \PY{n}{t} \PY{o}{=} \PY{l+m+mf}{0.0}
         \PY{n}{tmax} \PY{o}{=} \PY{l+m+mi}{30}
             
         \PY{k}{def} \PY{n+nf}{calcNextW}\PY{p}{(}\PY{n}{innerz}\PY{p}{,} \PY{n}{inneromega}\PY{p}{)}\PY{p}{:}
             \PY{n}{dz} \PY{o}{=}   \PY{p}{[}\PY{n}{innerz}\PY{p}{[}\PY{o}{\PYZhy{}}\PY{l+m+mi}{1}\PY{p}{]}\PY{p}{[}\PY{l+m+mi}{1}\PY{p}{]}\PY{p}{,} \PY{l+m+mi}{1} \PY{o}{/} \PY{n}{m} \PY{o}{*} \PY{p}{(}\PY{n}{F0} \PY{o}{*} \PY{n}{math}\PY{o}{.}\PY{n}{cos}\PY{p}{(}\PY{n}{inneromega} \PY{o}{*} \PY{n}{t}\PY{p}{)} \PY{o}{\PYZhy{}} \PY{n}{D} \PY{o}{*} \PY{n}{innerz}\PY{p}{[}\PY{o}{\PYZhy{}}\PY{l+m+mi}{1}\PY{p}{]}\PY{p}{[}\PY{l+m+mi}{0}\PY{p}{]} \PY{o}{\PYZhy{}} \PY{n}{D3} \PY{o}{*} \PY{n}{innerz}\PY{p}{[}\PY{o}{\PYZhy{}}\PY{l+m+mi}{1}\PY{p}{]}\PY{p}{[}\PY{l+m+mi}{0}\PY{p}{]}\PY{o}{*}\PY{o}{*}\PY{l+m+mi}{3} \PY{o}{\PYZhy{}} \PY{n}{c} \PY{o}{*} \PY{n}{innerz}\PY{p}{[}\PY{o}{\PYZhy{}}\PY{l+m+mi}{1}\PY{p}{]}\PY{p}{[}\PY{l+m+mi}{1}\PY{p}{]} \PY{p}{)}\PY{p}{]}
             \PY{n}{znew} \PY{o}{=} \PY{p}{[}\PY{n}{innerz}\PY{p}{[}\PY{o}{\PYZhy{}}\PY{l+m+mi}{1}\PY{p}{]}\PY{p}{[}\PY{l+m+mi}{0}\PY{p}{]} \PY{o}{+} \PY{n}{dz}\PY{p}{[}\PY{l+m+mi}{0}\PY{p}{]} \PY{o}{*} \PY{n}{dt} \PY{p}{,} \PY{n}{innerz}\PY{p}{[}\PY{o}{\PYZhy{}}\PY{l+m+mi}{1}\PY{p}{]}\PY{p}{[}\PY{l+m+mi}{1}\PY{p}{]} \PY{o}{+} \PY{n}{dz}\PY{p}{[}\PY{l+m+mi}{1}\PY{p}{]} \PY{o}{*} \PY{n}{dt}\PY{p}{]}
             \PY{n}{innerz}\PY{o}{.}\PY{n}{append}\PY{p}{(}\PY{n}{znew}\PY{p}{)}
         
         \PY{k}{while} \PY{n}{t} \PY{o}{\PYZlt{}} \PY{n}{tmax}\PY{p}{:}
             \PY{n}{t} \PY{o}{=} \PY{n}{t} \PY{o}{+} \PY{n}{dt}
             \PY{n}{calcNextW}\PY{p}{(}\PY{n}{z1}\PY{p}{,} \PY{n}{omega1}\PY{p}{)}\PY{p}{;}
         
         \PY{n}{plt}\PY{o}{.}\PY{n}{title}\PY{p}{(}\PY{l+s+s1}{\PYZsq{}}\PY{l+s+s1}{Zeit\PYZhy{}Weg\PYZhy{}Diagramm}\PY{l+s+s1}{\PYZsq{}}\PY{p}{)}
         \PY{n}{plt}\PY{o}{.}\PY{n}{xlabel}\PY{p}{(}\PY{l+s+s1}{\PYZsq{}}\PY{l+s+s1}{Zeit (t)}\PY{l+s+s1}{\PYZsq{}}\PY{p}{)}
         \PY{n}{plt}\PY{o}{.}\PY{n}{ylabel}\PY{p}{(}\PY{l+s+s1}{\PYZsq{}}\PY{l+s+s1}{Amplitude}\PY{l+s+s1}{\PYZsq{}}\PY{p}{)}
         \PY{n}{plt}\PY{o}{.}\PY{n}{plot}\PY{p}{(}\PY{n+nb}{range}\PY{p}{(}\PY{l+m+mi}{0}\PY{p}{,}\PY{n+nb}{len}\PY{p}{(}\PY{n}{z1}\PY{p}{)}\PY{p}{)}\PY{p}{,} \PY{p}{[}\PY{n}{element}\PY{p}{[}\PY{l+m+mi}{0}\PY{p}{]} \PY{k}{for} \PY{n}{element} \PY{o+ow}{in} \PY{n}{z1}\PY{p}{]}\PY{p}{)}
\end{Verbatim}


\begin{Verbatim}[commandchars=\\\{\}]
{\color{outcolor}Out[{\color{outcolor}13}]:} [<matplotlib.lines.Line2D at 0x115966780>]
\end{Verbatim}
            
    \begin{center}
    \adjustimage{max size={0.9\linewidth}{0.9\paperheight}}{output_29_1.png}
    \end{center}
    { \hspace*{\fill} \\}
    
    Es ist zu sehen, dass sich die bisherige Schwingung mit dem
hinzugefügten Teil überlagert.

    \subsection{Aufgabe 4: Das nichtlineare Pendel
II}\label{aufgabe-4-das-nichtlineare-pendel-ii}

    \begin{Verbatim}[commandchars=\\\{\}]
{\color{incolor}In [{\color{incolor}14}]:} \PY{n}{z1}\PY{o}{=}\PY{p}{[}\PY{p}{[}\PY{l+m+mi}{1}\PY{p}{,}\PY{l+m+mi}{0}\PY{p}{]}\PY{p}{]}
         \PY{n}{z2}\PY{o}{=}\PY{p}{[}\PY{p}{[}\PY{l+m+mi}{1}\PY{p}{,}\PY{l+m+mi}{0}\PY{p}{]}\PY{p}{]}
         \PY{n}{z3}\PY{o}{=}\PY{p}{[}\PY{p}{[}\PY{l+m+mi}{1}\PY{p}{,}\PY{l+m+mi}{0}\PY{p}{]}\PY{p}{]}
         \PY{n}{z4}\PY{o}{=}\PY{p}{[}\PY{p}{[}\PY{l+m+mi}{1}\PY{p}{,}\PY{l+m+mi}{0}\PY{p}{]}\PY{p}{]}
         
         \PY{n}{dt} \PY{o}{=} \PY{l+m+mf}{0.001}
         
         \PY{n}{F1} \PY{o}{=} \PY{l+m+mf}{0.1}
         \PY{n}{F2} \PY{o}{=} \PY{l+m+mf}{0.95}
         \PY{n}{F3} \PY{o}{=} \PY{l+m+mi}{1}
         \PY{n}{F4} \PY{o}{=} \PY{l+m+mf}{1.5}
         
         \PY{n}{t} \PY{o}{=} \PY{l+m+mf}{0.0}
         \PY{n}{tmax} \PY{o}{=} \PY{l+m+mi}{300}
             
         \PY{k}{def} \PY{n+nf}{calcNext}\PY{p}{(}\PY{n}{zinner}\PY{p}{,} \PY{n}{Finner}\PY{p}{)}\PY{p}{:}
             \PY{n}{dz} \PY{o}{=}   \PY{p}{[}\PY{n}{zinner}\PY{p}{[}\PY{o}{\PYZhy{}}\PY{l+m+mi}{1}\PY{p}{]}\PY{p}{[}\PY{l+m+mi}{1}\PY{p}{]}\PY{p}{,} \PY{n}{Finner} \PY{o}{*} \PY{n}{math}\PY{o}{.}\PY{n}{cos}\PY{p}{(}\PY{l+m+mf}{0.6} \PY{o}{*} \PY{n}{t}\PY{p}{)} \PY{o}{\PYZhy{}} \PY{l+m+mf}{0.5} \PY{o}{*} \PY{n}{zinner}\PY{p}{[}\PY{o}{\PYZhy{}}\PY{l+m+mi}{1}\PY{p}{]}\PY{p}{[}\PY{l+m+mi}{1}\PY{p}{]} \PY{o}{\PYZhy{}} \PY{n}{math}\PY{o}{.}\PY{n}{sin}\PY{p}{(}\PY{n}{zinner}\PY{p}{[}\PY{o}{\PYZhy{}}\PY{l+m+mi}{1}\PY{p}{]}\PY{p}{[}\PY{l+m+mi}{0}\PY{p}{]}\PY{p}{)}\PY{p}{]}
             \PY{n}{znew} \PY{o}{=} \PY{p}{[}\PY{n}{zinner}\PY{p}{[}\PY{o}{\PYZhy{}}\PY{l+m+mi}{1}\PY{p}{]}\PY{p}{[}\PY{l+m+mi}{0}\PY{p}{]} \PY{o}{+} \PY{n}{dz}\PY{p}{[}\PY{l+m+mi}{0}\PY{p}{]} \PY{o}{*} \PY{n}{dt} \PY{p}{,} \PY{n}{zinner}\PY{p}{[}\PY{o}{\PYZhy{}}\PY{l+m+mi}{1}\PY{p}{]}\PY{p}{[}\PY{l+m+mi}{1}\PY{p}{]} \PY{o}{+} \PY{n}{dz}\PY{p}{[}\PY{l+m+mi}{1}\PY{p}{]} \PY{o}{*} \PY{n}{dt}\PY{p}{]}
             \PY{n}{zinner}\PY{o}{.}\PY{n}{append}\PY{p}{(}\PY{n}{znew}\PY{p}{)}
             
         
         \PY{k}{while} \PY{n}{t} \PY{o}{\PYZlt{}} \PY{n}{tmax}\PY{p}{:}
             \PY{n}{t} \PY{o}{=} \PY{n}{t} \PY{o}{+} \PY{n}{dt}
             \PY{n}{calcNext}\PY{p}{(}\PY{n}{z1}\PY{p}{,} \PY{n}{F1}\PY{p}{)}\PY{p}{;}
             \PY{n}{calcNext}\PY{p}{(}\PY{n}{z2}\PY{p}{,} \PY{n}{F2}\PY{p}{)}\PY{p}{;}
             \PY{n}{calcNext}\PY{p}{(}\PY{n}{z3}\PY{p}{,} \PY{n}{F3}\PY{p}{)}\PY{p}{;}
             \PY{n}{calcNext}\PY{p}{(}\PY{n}{z4}\PY{p}{,} \PY{n}{F4}\PY{p}{)}\PY{p}{;}
             
         \PY{n}{plt}\PY{o}{.}\PY{n}{plot}\PY{p}{(}\PY{n+nb}{range}\PY{p}{(}\PY{l+m+mi}{0}\PY{p}{,}\PY{n+nb}{len}\PY{p}{(}\PY{n}{z1}\PY{p}{)}\PY{p}{)}\PY{p}{,} \PY{p}{[}\PY{n}{element}\PY{p}{[}\PY{l+m+mi}{0}\PY{p}{]} \PY{k}{for} \PY{n}{element} \PY{o+ow}{in} \PY{n}{z1}\PY{p}{]}\PY{p}{,} \PY{n}{label}\PY{o}{=}\PY{l+s+s1}{\PYZsq{}}\PY{l+s+s1}{F=}\PY{l+s+s1}{\PYZsq{}} \PY{o}{+} \PY{n+nb}{str}\PY{p}{(}\PY{n}{F1}\PY{p}{)}\PY{p}{)}
         \PY{n}{plt}\PY{o}{.}\PY{n}{plot}\PY{p}{(}\PY{n+nb}{range}\PY{p}{(}\PY{l+m+mi}{0}\PY{p}{,}\PY{n+nb}{len}\PY{p}{(}\PY{n}{z2}\PY{p}{)}\PY{p}{)}\PY{p}{,} \PY{p}{[}\PY{n}{element}\PY{p}{[}\PY{l+m+mi}{0}\PY{p}{]} \PY{k}{for} \PY{n}{element} \PY{o+ow}{in} \PY{n}{z2}\PY{p}{]}\PY{p}{,} \PY{n}{label}\PY{o}{=}\PY{l+s+s1}{\PYZsq{}}\PY{l+s+s1}{F=}\PY{l+s+s1}{\PYZsq{}} \PY{o}{+} \PY{n+nb}{str}\PY{p}{(}\PY{n}{F2}\PY{p}{)}\PY{p}{)}
         \PY{n}{plt}\PY{o}{.}\PY{n}{plot}\PY{p}{(}\PY{n+nb}{range}\PY{p}{(}\PY{l+m+mi}{0}\PY{p}{,}\PY{n+nb}{len}\PY{p}{(}\PY{n}{z3}\PY{p}{)}\PY{p}{)}\PY{p}{,} \PY{p}{[}\PY{n}{element}\PY{p}{[}\PY{l+m+mi}{0}\PY{p}{]} \PY{k}{for} \PY{n}{element} \PY{o+ow}{in} \PY{n}{z3}\PY{p}{]}\PY{p}{,} \PY{n}{label}\PY{o}{=}\PY{l+s+s1}{\PYZsq{}}\PY{l+s+s1}{F=}\PY{l+s+s1}{\PYZsq{}} \PY{o}{+} \PY{n+nb}{str}\PY{p}{(}\PY{n}{F3}\PY{p}{)}\PY{p}{)}
         \PY{n}{plt}\PY{o}{.}\PY{n}{plot}\PY{p}{(}\PY{n+nb}{range}\PY{p}{(}\PY{l+m+mi}{0}\PY{p}{,}\PY{n+nb}{len}\PY{p}{(}\PY{n}{z4}\PY{p}{)}\PY{p}{)}\PY{p}{,} \PY{p}{[}\PY{n}{element}\PY{p}{[}\PY{l+m+mi}{0}\PY{p}{]} \PY{k}{for} \PY{n}{element} \PY{o+ow}{in} \PY{n}{z4}\PY{p}{]}\PY{p}{,} \PY{n}{label}\PY{o}{=}\PY{l+s+s1}{\PYZsq{}}\PY{l+s+s1}{F=}\PY{l+s+s1}{\PYZsq{}} \PY{o}{+} \PY{n+nb}{str}\PY{p}{(}\PY{n}{F4}\PY{p}{)}\PY{p}{)}
         
         \PY{n}{plt}\PY{o}{.}\PY{n}{title}\PY{p}{(}\PY{l+s+s1}{\PYZsq{}}\PY{l+s+s1}{Zeit\PYZhy{}Weg\PYZhy{}Diagramm}\PY{l+s+s1}{\PYZsq{}}\PY{p}{)}
         \PY{n}{plt}\PY{o}{.}\PY{n}{xlabel}\PY{p}{(}\PY{l+s+s1}{\PYZsq{}}\PY{l+s+s1}{Zeit (t)}\PY{l+s+s1}{\PYZsq{}}\PY{p}{)}
         \PY{n}{plt}\PY{o}{.}\PY{n}{ylabel}\PY{p}{(}\PY{l+s+s1}{\PYZsq{}}\PY{l+s+s1}{Amplitude}\PY{l+s+s1}{\PYZsq{}}\PY{p}{)}
         \PY{n}{plt}\PY{o}{.}\PY{n}{legend}\PY{p}{(}\PY{p}{)}
\end{Verbatim}


\begin{Verbatim}[commandchars=\\\{\}]
{\color{outcolor}Out[{\color{outcolor}14}]:} <matplotlib.legend.Legend at 0x1529f34978>
\end{Verbatim}
            
    \begin{center}
    \adjustimage{max size={0.9\linewidth}{0.9\paperheight}}{output_32_1.png}
    \end{center}
    { \hspace*{\fill} \\}
    
    Für die erste Schwingung (blau) ist zu erkennen, dass sie sehr schnell
stabil wird. Das "Einschwingen" ist hier nicht zu erkennen. Auch die
zweite Kurve (orange) wird nach einiger Zeit stabil. Hier lässt sich
jedoch das "Einschwingen" beobachten.

Die beiden anderen Schwingungen erreichen keinen stabilen Zustand. Dies
lässt sich daran erkennen, dass in deren Kurve keine Wiederholung zu
sehen ist.

    Als nächstes wird θ(θ') gebildet. Da der Zustandsvektor bereits den Ort
θ sowie die Geschwindigkeit θ' enthällt, ist keine weitere numerische
Berechnung notwendig. Die Graphen können direkt gezeichnet werden.

    \begin{Verbatim}[commandchars=\\\{\}]
{\color{incolor}In [{\color{incolor}15}]:} \PY{n}{plt}\PY{o}{.}\PY{n}{plot}\PY{p}{(}\PY{p}{[}\PY{n}{element}\PY{p}{[}\PY{l+m+mi}{0}\PY{p}{]} \PY{k}{for} \PY{n}{element} \PY{o+ow}{in} \PY{n}{z1}\PY{p}{]}\PY{p}{,} \PY{p}{[}\PY{n}{element}\PY{p}{[}\PY{l+m+mi}{1}\PY{p}{]} \PY{k}{for} \PY{n}{element} \PY{o+ow}{in} \PY{n}{z1}\PY{p}{]}\PY{p}{)}
         
         \PY{n}{plt}\PY{o}{.}\PY{n}{title}\PY{p}{(}\PY{l+s+s1}{\PYZsq{}}\PY{l+s+s1}{Phasenraum}\PY{l+s+s1}{\PYZsq{}}\PY{p}{)}
         \PY{n}{plt}\PY{o}{.}\PY{n}{xlabel}\PY{p}{(}\PY{l+s+s2}{\PYZdq{}}\PY{l+s+s2}{θ}\PY{l+s+s2}{\PYZdq{}}\PY{p}{)}
         \PY{n}{plt}\PY{o}{.}\PY{n}{ylabel}\PY{p}{(}\PY{l+s+s2}{\PYZdq{}}\PY{l+s+s2}{θ}\PY{l+s+s2}{\PYZsq{}}\PY{l+s+s2}{(θ)}\PY{l+s+s2}{\PYZdq{}}\PY{p}{)}
\end{Verbatim}


\begin{Verbatim}[commandchars=\\\{\}]
{\color{outcolor}Out[{\color{outcolor}15}]:} Text(0,0.5,"θ'(θ)")
\end{Verbatim}
            
    \begin{center}
    \adjustimage{max size={0.9\linewidth}{0.9\paperheight}}{output_35_1.png}
    \end{center}
    { \hspace*{\fill} \\}
    
    Wie bereits in oben stehender Graphik zu erkennen ist, erreicht die
erste Schwingung äußerst schnell einen stabilen Zustand. (zu erkennen an
der Schnellen Konvergenz)

    \begin{Verbatim}[commandchars=\\\{\}]
{\color{incolor}In [{\color{incolor}16}]:} \PY{n}{plt}\PY{o}{.}\PY{n}{plot}\PY{p}{(}\PY{p}{[}\PY{n}{element}\PY{p}{[}\PY{l+m+mi}{0}\PY{p}{]} \PY{k}{for} \PY{n}{element} \PY{o+ow}{in} \PY{n}{z2}\PY{p}{]}\PY{p}{,} \PY{p}{[}\PY{n}{element}\PY{p}{[}\PY{l+m+mi}{1}\PY{p}{]} \PY{k}{for} \PY{n}{element} \PY{o+ow}{in} \PY{n}{z2}\PY{p}{]}\PY{p}{)}
         
         \PY{n}{plt}\PY{o}{.}\PY{n}{title}\PY{p}{(}\PY{l+s+s1}{\PYZsq{}}\PY{l+s+s1}{Phasenraum}\PY{l+s+s1}{\PYZsq{}}\PY{p}{)}
         \PY{n}{plt}\PY{o}{.}\PY{n}{xlabel}\PY{p}{(}\PY{l+s+s2}{\PYZdq{}}\PY{l+s+s2}{θ}\PY{l+s+s2}{\PYZdq{}}\PY{p}{)}
         \PY{n}{plt}\PY{o}{.}\PY{n}{ylabel}\PY{p}{(}\PY{l+s+s2}{\PYZdq{}}\PY{l+s+s2}{θ}\PY{l+s+s2}{\PYZsq{}}\PY{l+s+s2}{(θ)}\PY{l+s+s2}{\PYZdq{}}\PY{p}{)}
\end{Verbatim}


\begin{Verbatim}[commandchars=\\\{\}]
{\color{outcolor}Out[{\color{outcolor}16}]:} Text(0,0.5,"θ'(θ)")
\end{Verbatim}
            
    \begin{center}
    \adjustimage{max size={0.9\linewidth}{0.9\paperheight}}{output_37_1.png}
    \end{center}
    { \hspace*{\fill} \\}
    
    Auch Schwingung 2 erreicht einen stabilen Zustand. Jedoch wird hierfür
mehr Zeit als bei Schwingung 1 benötigt.

    \begin{Verbatim}[commandchars=\\\{\}]
{\color{incolor}In [{\color{incolor}17}]:} \PY{n}{plt}\PY{o}{.}\PY{n}{plot}\PY{p}{(}\PY{p}{[}\PY{n}{element}\PY{p}{[}\PY{l+m+mi}{0}\PY{p}{]} \PY{k}{for} \PY{n}{element} \PY{o+ow}{in} \PY{n}{z3}\PY{p}{]}\PY{p}{,} \PY{p}{[}\PY{n}{element}\PY{p}{[}\PY{l+m+mi}{1}\PY{p}{]} \PY{k}{for} \PY{n}{element} \PY{o+ow}{in} \PY{n}{z3}\PY{p}{]}\PY{p}{)}
         
         \PY{n}{plt}\PY{o}{.}\PY{n}{title}\PY{p}{(}\PY{l+s+s1}{\PYZsq{}}\PY{l+s+s1}{Phasenraum}\PY{l+s+s1}{\PYZsq{}}\PY{p}{)}
         \PY{n}{plt}\PY{o}{.}\PY{n}{xlabel}\PY{p}{(}\PY{l+s+s2}{\PYZdq{}}\PY{l+s+s2}{θ}\PY{l+s+s2}{\PYZdq{}}\PY{p}{)}
         \PY{n}{plt}\PY{o}{.}\PY{n}{ylabel}\PY{p}{(}\PY{l+s+s2}{\PYZdq{}}\PY{l+s+s2}{θ}\PY{l+s+s2}{\PYZsq{}}\PY{l+s+s2}{(θ)}\PY{l+s+s2}{\PYZdq{}}\PY{p}{)}
\end{Verbatim}


\begin{Verbatim}[commandchars=\\\{\}]
{\color{outcolor}Out[{\color{outcolor}17}]:} Text(0,0.5,"θ'(θ)")
\end{Verbatim}
            
    \begin{center}
    \adjustimage{max size={0.9\linewidth}{0.9\paperheight}}{output_39_1.png}
    \end{center}
    { \hspace*{\fill} \\}
    
    \begin{Verbatim}[commandchars=\\\{\}]
{\color{incolor}In [{\color{incolor}18}]:} \PY{n}{plt}\PY{o}{.}\PY{n}{plot}\PY{p}{(}\PY{p}{[}\PY{n}{element}\PY{p}{[}\PY{l+m+mi}{0}\PY{p}{]} \PY{k}{for} \PY{n}{element} \PY{o+ow}{in} \PY{n}{z4}\PY{p}{]}\PY{p}{,} \PY{p}{[}\PY{n}{element}\PY{p}{[}\PY{l+m+mi}{1}\PY{p}{]} \PY{k}{for} \PY{n}{element} \PY{o+ow}{in} \PY{n}{z4}\PY{p}{]}\PY{p}{)}
         
         \PY{n}{plt}\PY{o}{.}\PY{n}{title}\PY{p}{(}\PY{l+s+s1}{\PYZsq{}}\PY{l+s+s1}{Phasenraum}\PY{l+s+s1}{\PYZsq{}}\PY{p}{)}
         \PY{n}{plt}\PY{o}{.}\PY{n}{xlabel}\PY{p}{(}\PY{l+s+s2}{\PYZdq{}}\PY{l+s+s2}{θ}\PY{l+s+s2}{\PYZdq{}}\PY{p}{)}
         \PY{n}{plt}\PY{o}{.}\PY{n}{ylabel}\PY{p}{(}\PY{l+s+s2}{\PYZdq{}}\PY{l+s+s2}{θ}\PY{l+s+s2}{\PYZsq{}}\PY{l+s+s2}{(θ)}\PY{l+s+s2}{\PYZdq{}}\PY{p}{)}
\end{Verbatim}


\begin{Verbatim}[commandchars=\\\{\}]
{\color{outcolor}Out[{\color{outcolor}18}]:} Text(0,0.5,"θ'(θ)")
\end{Verbatim}
            
    \begin{center}
    \adjustimage{max size={0.9\linewidth}{0.9\paperheight}}{output_40_1.png}
    \end{center}
    { \hspace*{\fill} \\}
    
    Auf dem Graph der letzen beiden Schwingungen ist keine Konvergenz zu
erkennen.


    % Add a bibliography block to the postdoc
    
    
    
    \end{document}
